%% Generated by Sphinx.
\def\sphinxdocclass{report}
\documentclass[letterpaper,10pt,openany,oneside,english]{sphinxmanual}
\ifdefined\pdfpxdimen
   \let\sphinxpxdimen\pdfpxdimen\else\newdimen\sphinxpxdimen
\fi \sphinxpxdimen=.75bp\relax

\PassOptionsToPackage{warn}{textcomp}
\usepackage[utf8]{inputenc}
\ifdefined\DeclareUnicodeCharacter
% support both utf8 and utf8x syntaxes
\edef\sphinxdqmaybe{\ifdefined\DeclareUnicodeCharacterAsOptional\string"\fi}
  \DeclareUnicodeCharacter{\sphinxdqmaybe00A0}{\nobreakspace}
  \DeclareUnicodeCharacter{\sphinxdqmaybe2500}{\sphinxunichar{2500}}
  \DeclareUnicodeCharacter{\sphinxdqmaybe2502}{\sphinxunichar{2502}}
  \DeclareUnicodeCharacter{\sphinxdqmaybe2514}{\sphinxunichar{2514}}
  \DeclareUnicodeCharacter{\sphinxdqmaybe251C}{\sphinxunichar{251C}}
  \DeclareUnicodeCharacter{\sphinxdqmaybe2572}{\textbackslash}
\fi
\usepackage{cmap}
\usepackage[T1]{fontenc}
\usepackage{amsmath,amssymb,amstext}
\usepackage[english]{babel}
\usepackage{times}
\usepackage[Bjarne]{fncychap}
\usepackage{sphinx}

\fvset{fontsize=\small}
\usepackage{geometry}

% Include hyperref last.
\usepackage{hyperref}
% Fix anchor placement for figures with captions.
\usepackage{hypcap}% it must be loaded after hyperref.
% Set up styles of URL: it should be placed after hyperref.
\urlstyle{same}

\addto\captionsenglish{\renewcommand{\figurename}{Fig.}}
\addto\captionsenglish{\renewcommand{\tablename}{Table}}
\addto\captionsenglish{\renewcommand{\literalblockname}{Listing}}

\addto\captionsenglish{\renewcommand{\literalblockcontinuedname}{continued from previous page}}
\addto\captionsenglish{\renewcommand{\literalblockcontinuesname}{continues on next page}}
\addto\captionsenglish{\renewcommand{\sphinxnonalphabeticalgroupname}{Non-alphabetical}}
\addto\captionsenglish{\renewcommand{\sphinxsymbolsname}{Symbols}}
\addto\captionsenglish{\renewcommand{\sphinxnumbersname}{Numbers}}

\addto\extrasenglish{\def\pageautorefname{page}}

\setcounter{tocdepth}{1}


% Jupyter Notebook prompt colors
\definecolor{nbsphinxin}{HTML}{303F9F}
\definecolor{nbsphinxout}{HTML}{D84315}
% ANSI colors for output streams and traceback highlighting
\definecolor{ansi-black}{HTML}{3E424D}
\definecolor{ansi-black-intense}{HTML}{282C36}
\definecolor{ansi-red}{HTML}{E75C58}
\definecolor{ansi-red-intense}{HTML}{B22B31}
\definecolor{ansi-green}{HTML}{00A250}
\definecolor{ansi-green-intense}{HTML}{007427}
\definecolor{ansi-yellow}{HTML}{DDB62B}
\definecolor{ansi-yellow-intense}{HTML}{B27D12}
\definecolor{ansi-blue}{HTML}{208FFB}
\definecolor{ansi-blue-intense}{HTML}{0065CA}
\definecolor{ansi-magenta}{HTML}{D160C4}
\definecolor{ansi-magenta-intense}{HTML}{A03196}
\definecolor{ansi-cyan}{HTML}{60C6C8}
\definecolor{ansi-cyan-intense}{HTML}{258F8F}
\definecolor{ansi-white}{HTML}{C5C1B4}
\definecolor{ansi-white-intense}{HTML}{A1A6B2}
\definecolor{ansi-default-inverse-fg}{HTML}{FFFFFF}
\definecolor{ansi-default-inverse-bg}{HTML}{000000}

% Define "notice" environment, which was removed in Sphinx 1.7.
% At some point, "notice" should be replaced by "sphinxadmonition",
% which is available since Sphinx 1.5.
\makeatletter
\@ifundefined{notice}{%
\newenvironment{notice}{\begin{sphinxadmonition}}{\end{sphinxadmonition}}%
}{}
\makeatother

\usepackage{enumitem}
\setlistdepth{99}

\title{fredpy Documentation}
\date{Nov 26, 2018}
\release{3.0.3}
\author{Brian C. Jenkins}
\newcommand{\sphinxlogo}{\vbox{}}
\renewcommand{\releasename}{Release}
\makeindex
\begin{document}

\pagestyle{empty}
\maketitle
\pagestyle{plain}
\sphinxtableofcontents
\pagestyle{normal}
\phantomsection\label{\detokenize{index::doc}}


\sphinxcode{\sphinxupquote{fredpy}} is a Python package for retrieving and working with data from Federal Reserve Economic Data (FRED). The package makes it easy to download specific data series and provides a set of tools for transforming the data in order to construct plots and do statistical analysis. The \sphinxcode{\sphinxupquote{fredpy}} package is useful for anyone doing empirical research using data from FRED and for anyone, e.g. economics teachers, students, and journalists, that will benefit from having an efficient and flexible way to access FRED with Python. \sphinxcode{\sphinxupquote{fredpy}} is compatible with Python 2 and 3.


\chapter{Installation}
\label{\detokenize{index:installation}}
Install \sphinxcode{\sphinxupquote{fredpy}} from PyPI with the shell command:

\fvset{hllines={, ,}}%
\begin{sphinxVerbatim}[commandchars=\\\{\}]
\PYG{n}{pip} \PYG{n}{install} \PYG{n}{fredpy}
\end{sphinxVerbatim}

Or download the source here: \sphinxurl{https://github.com/letsgoexploring/fredpy-package/raw/gh-pages/dist/fredpy-3.0.2.tar.gz}


\chapter{Contents:}
\label{\detokenize{index:contents}}

\section{\sphinxstyleliteralintitle{\sphinxupquote{fredpy.series}} class}
\label{\detokenize{series_class:fredpy-series-class}}\label{\detokenize{series_class::doc}}\index{fredpy.series (built-in class)@\spxentry{fredpy.series}\spxextra{built-in class}}

\begin{fulllineitems}
\phantomsection\label{\detokenize{series_class:fredpy.series}}\pysiglinewithargsret{\sphinxbfcode{\sphinxupquote{class }}\sphinxcode{\sphinxupquote{fredpy.}}\sphinxbfcode{\sphinxupquote{series}}}{\emph{series\_id=None}}{}
Creates an instance of a {\hyperref[\detokenize{series_class:fredpy.series}]{\sphinxcrossref{\sphinxcode{\sphinxupquote{fredpy.series}}}}} instance that stores information about the specified data series from FRED with the unique series ID code given by \sphinxcode{\sphinxupquote{series\_id}}.
\begin{quote}\begin{description}
\item[{Parameters}] \leavevmode
\sphinxstyleliteralstrong{\sphinxupquote{series\_id}} (\sphinxstyleliteralemphasis{\sphinxupquote{string}}) \textendash{} unique FRED series ID. If \sphinxcode{\sphinxupquote{series\_id}} equals None, an empty {\hyperref[\detokenize{series_class:fredpy.series}]{\sphinxcrossref{\sphinxcode{\sphinxupquote{fredpy.series}}}}} instance is created.

\end{description}\end{quote}

\sphinxstylestrong{Attributes:}
\begin{quote}
\begin{quote}\begin{description}
\item[{data}] \leavevmode
(numpy ndarray) \textendash{}  data values.

\item[{daterange}] \leavevmode
(string) \textendash{} specifies the dates of the first and last observations.

\item[{dates}] \leavevmode
(list) \textendash{} list of date strings in YYYY-MM-DD format.

\item[{datetimes}] \leavevmode
(numpy ndarray) \textendash{} array containing observation dates formatted as \sphinxhref{https://docs.python.org/3/library/datetime.html\#datetime.datetime}{\sphinxcode{\sphinxupquote{datetime.datetime}}} instances.

\item[{frequency}] \leavevmode
(string) \textendash{} data frequency. ‘Daily’, ‘Weekly’, ‘Monthly’, ‘Quarterly’, or ‘Annual’.

\item[{frequency\_short}] \leavevmode
(string) \textendash{} data frequency. Abbreviated. ‘D’, ‘W’, ‘M’, ‘Q’, ‘SA, or ‘A’.

\item[{last\_updated}] \leavevmode
(string) \textendash{} date series was last updated.

\item[{notes}] \leavevmode
(string) \textendash{} details about series. Not available for all series.

\item[{release}] \leavevmode
(string) \textendash{} statistical release containing data.

\item[{seasonal\_adjustment}] \leavevmode
(string) \textendash{} specifies whether the data has been seasonally adjusted.

\item[{seasonal\_adjustment\_short}] \leavevmode
(string) \textendash{}specifies whether the data has been seasonally adjusted. Abbreviated.

\item[{series\_id}] \leavevmode
(string) \textendash{} unique FRED series ID code.

\item[{source}] \leavevmode
(string) \textendash{} original source of the data.

\item[{t}] \leavevmode
(integer) \textendash{} number corresponding to frequency: 365 for daily, 52 for weekly, 12 for monthly, 4 for quarterly, and 1 for annual.

\item[{title}] \leavevmode
(string) \textendash{} title of the data series.

\item[{units}] \leavevmode
(string) \textendash{} units of the data series.

\item[{units\_short}] \leavevmode
(string) units of the data series. Abbreviated.

\end{description}\end{quote}
\end{quote}

\sphinxstylestrong{Methods:}
\begin{quote}
\index{fredpy.series.apc() (built-in function)@\spxentry{fredpy.series.apc()}\spxextra{built-in function}}

\begin{fulllineitems}
\phantomsection\label{\detokenize{series_class:fredpy.series.apc}}\pysiglinewithargsret{\sphinxbfcode{\sphinxupquote{apc}}}{\emph{log=True}, \emph{method='backward'}}{}
Computes the percentage change in the data over one year.
\begin{quote}\begin{description}
\item[{Parameters}] \leavevmode\begin{itemize}
\item {} 
\sphinxstyleliteralstrong{\sphinxupquote{log}} (\sphinxhref{https://docs.python.org/3/library/functions.html\#bool}{\sphinxstyleliteralemphasis{\sphinxupquote{bool}}}) \textendash{} If True, computes the percentage change as \(100\cdot\log(x_{t}/x_{t-1})\). If False, compute the percentage change as \(100\cdot\left( x_{t}/x_{t-1} - 1\right)\).

\item {} 
\sphinxstyleliteralstrong{\sphinxupquote{method}} (\sphinxstyleliteralemphasis{\sphinxupquote{string}}) \textendash{} If ‘backward’, compute percentage change from the previous period. If ‘forward’, compute percentage change from current to subsequent period.

\end{itemize}

\item[{Returns}] \leavevmode
{\hyperref[\detokenize{series_class:fredpy.series}]{\sphinxcrossref{\sphinxcode{\sphinxupquote{fredpy.series}}}}}

\end{description}\end{quote}

\end{fulllineitems}

\index{fredpy.series.bpfilter() (built-in function)@\spxentry{fredpy.series.bpfilter()}\spxextra{built-in function}}

\begin{fulllineitems}
\phantomsection\label{\detokenize{series_class:fredpy.series.bpfilter}}\pysiglinewithargsret{\sphinxbfcode{\sphinxupquote{bpfilter}}}{\emph{low=6}, \emph{high=32}, \emph{K=12}}{}
Computes the bandpass (Baxter-King) filter of the data. Returns a list of two {\hyperref[\detokenize{series_class:fredpy.series}]{\sphinxcrossref{\sphinxcode{\sphinxupquote{fredpy.series}}}}} instances containing the cyclical and trend components of the data:
\begin{quote}

\sphinxstyleemphasis{{[}new\_series\_cycle, new\_series\_trend{]}}
\end{quote}
\begin{quote}\begin{description}
\item[{Parameters}] \leavevmode\begin{itemize}
\item {} 
\sphinxstyleliteralstrong{\sphinxupquote{low}} (\sphinxhref{https://docs.python.org/3/library/functions.html\#int}{\sphinxstyleliteralemphasis{\sphinxupquote{int}}}) \textendash{} Minimum period for oscillations. Select 24 for monthly data, 6 for quarterly data (default), and 3 for annual data.

\item {} 
\sphinxstyleliteralstrong{\sphinxupquote{high}} (\sphinxhref{https://docs.python.org/3/library/functions.html\#int}{\sphinxstyleliteralemphasis{\sphinxupquote{int}}}) \textendash{} Maximum period for oscillations. Select 84 for monthly data, 32 for quarterly data (default), and 8 for annual data.

\item {} 
\sphinxstyleliteralstrong{\sphinxupquote{K}} (\sphinxhref{https://docs.python.org/3/library/functions.html\#int}{\sphinxstyleliteralemphasis{\sphinxupquote{int}}}) \textendash{} Lead-lag length of the filter. Select, 84 for monthly data, 12 for for quarterly data (default), and 1.5 for annual data.

\end{itemize}

\item[{Returns}] \leavevmode
\sphinxhref{https://docs.python.org/3/library/stdtypes.html\#list}{\sphinxcode{\sphinxupquote{list}}} of two {\hyperref[\detokenize{series_class:fredpy.series}]{\sphinxcrossref{\sphinxcode{\sphinxupquote{fredpy.series}}}}} instances

\end{description}\end{quote}

\begin{sphinxadmonition}{note}{Note:}
In computing the bandpass filter, K observations are lost from each end of the original series so the attributes \sphinxstyleemphasis{dates}, \sphinxstyleemphasis{datetimes}, and \sphinxstyleemphasis{data} are 2K elements shorter than their counterparts in the original series.
\end{sphinxadmonition}

\end{fulllineitems}

\index{fredpy.series.cffilter() (built-in function)@\spxentry{fredpy.series.cffilter()}\spxextra{built-in function}}

\begin{fulllineitems}
\phantomsection\label{\detokenize{series_class:fredpy.series.cffilter}}\pysiglinewithargsret{\sphinxbfcode{\sphinxupquote{cffilter}}}{\emph{low=6}, \emph{high=32}}{}
Computes the Christiano-Fitzgerald filter of the data. Returns a list of two {\hyperref[\detokenize{series_class:fredpy.series}]{\sphinxcrossref{\sphinxcode{\sphinxupquote{fredpy.series}}}}} instances containing the cyclical and trend components of the data:
\begin{quote}

\sphinxstyleemphasis{{[}new\_series\_cycle, new\_series\_trend{]}}
\end{quote}
\begin{quote}\begin{description}
\item[{Parameters}] \leavevmode\begin{itemize}
\item {} 
\sphinxstyleliteralstrong{\sphinxupquote{low}} (\sphinxhref{https://docs.python.org/3/library/functions.html\#int}{\sphinxstyleliteralemphasis{\sphinxupquote{int}}}) \textendash{} Minimum period for oscillations. Select 6 for quarterly data (default) and 1.5 for annual data.

\item {} 
\sphinxstyleliteralstrong{\sphinxupquote{high}} (\sphinxhref{https://docs.python.org/3/library/functions.html\#int}{\sphinxstyleliteralemphasis{\sphinxupquote{int}}}) \textendash{} Maximum period for oscillations. Select 32 for quarterly data (default) and 8 for annual data.

\end{itemize}

\item[{Returns}] \leavevmode
\sphinxhref{https://docs.python.org/3/library/stdtypes.html\#list}{\sphinxcode{\sphinxupquote{list}}} of two {\hyperref[\detokenize{series_class:fredpy.series}]{\sphinxcrossref{\sphinxcode{\sphinxupquote{fredpy.series}}}}} instances

\end{description}\end{quote}

\end{fulllineitems}

\index{fredpy.series.copy() (built-in function)@\spxentry{fredpy.series.copy()}\spxextra{built-in function}}

\begin{fulllineitems}
\phantomsection\label{\detokenize{series_class:fredpy.series.copy}}\pysiglinewithargsret{\sphinxbfcode{\sphinxupquote{copy}}}{}{}
Returns a copy of the {\hyperref[\detokenize{series_class:fredpy.series}]{\sphinxcrossref{\sphinxcode{\sphinxupquote{fredpy.series}}}}} instance.
\begin{quote}\begin{description}
\item[{Parameters}] \leavevmode
None

\item[{Returns}] \leavevmode
{\hyperref[\detokenize{series_class:fredpy.series}]{\sphinxcrossref{\sphinxcode{\sphinxupquote{fredpy.series}}}}}

\end{description}\end{quote}

\end{fulllineitems}

\index{fredpy.series.divide() (built-in function)@\spxentry{fredpy.series.divide()}\spxextra{built-in function}}

\begin{fulllineitems}
\phantomsection\label{\detokenize{series_class:fredpy.series.divide}}\pysiglinewithargsret{\sphinxbfcode{\sphinxupquote{divide}}}{\emph{series2}}{}
Divides the data from the current fredpy series by the data from \sphinxcode{\sphinxupquote{series2}}.
\begin{quote}\begin{description}
\item[{Parameters}] \leavevmode
\sphinxstyleliteralstrong{\sphinxupquote{series2}} ({\hyperref[\detokenize{series_class:fredpy.series}]{\sphinxcrossref{\sphinxstyleliteralemphasis{\sphinxupquote{fredpy.series}}}}}) \textendash{} A {\hyperref[\detokenize{series_class:fredpy.series}]{\sphinxcrossref{\sphinxcode{\sphinxupquote{fredpy.series}}}}} instance.

\item[{Returns}] \leavevmode
{\hyperref[\detokenize{series_class:fredpy.series}]{\sphinxcrossref{\sphinxcode{\sphinxupquote{fredpy.series}}}}}

\end{description}\end{quote}

\end{fulllineitems}

\index{fredpy.series.firstdiff() (built-in function)@\spxentry{fredpy.series.firstdiff()}\spxextra{built-in function}}

\begin{fulllineitems}
\phantomsection\label{\detokenize{series_class:fredpy.series.firstdiff}}\pysiglinewithargsret{\sphinxbfcode{\sphinxupquote{firstdiff}}}{}{}
Computes the first difference filter of original series. Returns a list of two {\hyperref[\detokenize{series_class:fredpy.series}]{\sphinxcrossref{\sphinxcode{\sphinxupquote{fredpy.series}}}}} instances containing the cyclical and trend components of the data:
\begin{quote}

\sphinxstyleemphasis{{[}new\_series\_cycle, new\_series\_trend{]}}
\end{quote}
\begin{quote}\begin{description}
\item[{Parameters}] \leavevmode
\item[{Returns}] \leavevmode
\sphinxhref{https://docs.python.org/3/library/stdtypes.html\#list}{\sphinxcode{\sphinxupquote{list}}} of two {\hyperref[\detokenize{series_class:fredpy.series}]{\sphinxcrossref{\sphinxcode{\sphinxupquote{fredpy.series}}}}} instances

\end{description}\end{quote}

\begin{sphinxadmonition}{note}{Note:}
In computing the first difference filter, the first observation from the original series is lost so the attributes \sphinxstyleemphasis{dates}, \sphinxstyleemphasis{datetimes}, and \sphinxstyleemphasis{data} are 1 element shorter than their counterparts in the original series.
\end{sphinxadmonition}

\end{fulllineitems}

\index{fredpy.series.hpfilter() (built-in function)@\spxentry{fredpy.series.hpfilter()}\spxextra{built-in function}}

\begin{fulllineitems}
\phantomsection\label{\detokenize{series_class:fredpy.series.hpfilter}}\pysiglinewithargsret{\sphinxbfcode{\sphinxupquote{hpfilter}}}{\emph{lamb=1600}}{}
Computes the Hodrick-Prescott filter of the data. Returns a list of two {\hyperref[\detokenize{series_class:fredpy.series}]{\sphinxcrossref{\sphinxcode{\sphinxupquote{fredpy.series}}}}} instances containing the cyclical and trend components of the data:
\begin{quote}

\sphinxstyleemphasis{{[}new\_series\_cycle, new\_series\_trend{]}}
\end{quote}
\begin{quote}\begin{description}
\item[{Parameters}] \leavevmode
\sphinxstyleliteralstrong{\sphinxupquote{lamb}} (\sphinxhref{https://docs.python.org/3/library/functions.html\#int}{\sphinxstyleliteralemphasis{\sphinxupquote{int}}}) \textendash{} The Hodrick-Prescott smoothing parameter. Select 129600 for monthly data, 1600 for quarterly data (default), and 6.25 for annual data.

\item[{Returns}] \leavevmode
\sphinxhref{https://docs.python.org/3/library/stdtypes.html\#list}{\sphinxcode{\sphinxupquote{list}}} of two {\hyperref[\detokenize{series_class:fredpy.series}]{\sphinxcrossref{\sphinxcode{\sphinxupquote{fredpy.series}}}}} instances

\end{description}\end{quote}

\end{fulllineitems}

\index{fredpy.series.lintrend() (built-in function)@\spxentry{fredpy.series.lintrend()}\spxextra{built-in function}}

\begin{fulllineitems}
\phantomsection\label{\detokenize{series_class:fredpy.series.lintrend}}\pysiglinewithargsret{\sphinxbfcode{\sphinxupquote{lintrend}}}{}{}
Computes a simple linear filter of the data using OLS. Returns a list of two {\hyperref[\detokenize{series_class:fredpy.series}]{\sphinxcrossref{\sphinxcode{\sphinxupquote{fredpy.series}}}}} instances containing the cyclical and trend components of the data:
\begin{quote}

\sphinxstyleemphasis{{[}new\_series\_cycle, new\_series\_trend{]}}
\end{quote}
\begin{quote}\begin{description}
\item[{Parameters}] \leavevmode
\item[{Returns}] \leavevmode
\sphinxhref{https://docs.python.org/3/library/stdtypes.html\#list}{\sphinxcode{\sphinxupquote{list}}} of two {\hyperref[\detokenize{series_class:fredpy.series}]{\sphinxcrossref{\sphinxcode{\sphinxupquote{fredpy.series}}}}} instances

\end{description}\end{quote}

\end{fulllineitems}

\index{fredpy.series.log() (built-in function)@\spxentry{fredpy.series.log()}\spxextra{built-in function}}

\begin{fulllineitems}
\phantomsection\label{\detokenize{series_class:fredpy.series.log}}\pysiglinewithargsret{\sphinxbfcode{\sphinxupquote{log}}}{}{}
Computes the natural log of the data.
\begin{quote}\begin{description}
\item[{Parameters}] \leavevmode
\item[{Returns}] \leavevmode
{\hyperref[\detokenize{series_class:fredpy.series}]{\sphinxcrossref{\sphinxcode{\sphinxupquote{fredpy.series}}}}}

\end{description}\end{quote}

\end{fulllineitems}

\index{fredpy.series.ma1side() (built-in function)@\spxentry{fredpy.series.ma1side()}\spxextra{built-in function}}

\begin{fulllineitems}
\phantomsection\label{\detokenize{series_class:fredpy.series.ma1side}}\pysiglinewithargsret{\sphinxbfcode{\sphinxupquote{ma1side}}}{\emph{length}}{}
Computes a one-sided moving average with window equal to \sphinxcode{\sphinxupquote{length}}.
\begin{quote}\begin{description}
\item[{Parameters}] \leavevmode
\sphinxstyleliteralstrong{\sphinxupquote{length}} (\sphinxhref{https://docs.python.org/3/library/functions.html\#int}{\sphinxstyleliteralemphasis{\sphinxupquote{int}}}) \textendash{} \sphinxcode{\sphinxupquote{length}} of the one-sided moving average.

\item[{Returns}] \leavevmode
{\hyperref[\detokenize{series_class:fredpy.series}]{\sphinxcrossref{\sphinxcode{\sphinxupquote{fredpy.series}}}}}

\end{description}\end{quote}

\end{fulllineitems}

\index{fredpy.series.ma2side() (built-in function)@\spxentry{fredpy.series.ma2side()}\spxextra{built-in function}}

\begin{fulllineitems}
\phantomsection\label{\detokenize{series_class:fredpy.series.ma2side}}\pysiglinewithargsret{\sphinxbfcode{\sphinxupquote{ma2side}}}{\emph{length}}{}
Computes a two-sided moving average with window equal to 2 times \sphinxcode{\sphinxupquote{length}}.
\begin{quote}\begin{description}
\item[{Parameters}] \leavevmode
\sphinxstyleliteralstrong{\sphinxupquote{length}} (\sphinxhref{https://docs.python.org/3/library/functions.html\#int}{\sphinxstyleliteralemphasis{\sphinxupquote{int}}}) \textendash{} half of \sphinxcode{\sphinxupquote{length}} of the two-sided moving average. For example, if \sphinxcode{\sphinxupquote{length = 12}}, then the moving average will contain 24 the 12 periods before and the 12 periods after each observation.

\item[{Returns}] \leavevmode
{\hyperref[\detokenize{series_class:fredpy.series}]{\sphinxcrossref{\sphinxcode{\sphinxupquote{fredpy.series}}}}}

\end{description}\end{quote}

\end{fulllineitems}

\index{fredpy.series.minus() (built-in function)@\spxentry{fredpy.series.minus()}\spxextra{built-in function}}

\begin{fulllineitems}
\phantomsection\label{\detokenize{series_class:fredpy.series.minus}}\pysiglinewithargsret{\sphinxbfcode{\sphinxupquote{minus}}}{\emph{series2}}{}
Subtracts the data from \sphinxcode{\sphinxupquote{series2}} from the data from the current fredpy series.
\begin{quote}\begin{description}
\item[{Parameters}] \leavevmode
\sphinxstyleliteralstrong{\sphinxupquote{series2}} ({\hyperref[\detokenize{series_class:fredpy.series}]{\sphinxcrossref{\sphinxstyleliteralemphasis{\sphinxupquote{fredpy.series}}}}}) \textendash{} A {\hyperref[\detokenize{series_class:fredpy.series}]{\sphinxcrossref{\sphinxcode{\sphinxupquote{fredpy.series}}}}} instance.

\item[{Returns}] \leavevmode
{\hyperref[\detokenize{series_class:fredpy.series}]{\sphinxcrossref{\sphinxcode{\sphinxupquote{fredpy.series}}}}}

\end{description}\end{quote}

\end{fulllineitems}

\index{fredpy.series.monthtoannual() (built-in function)@\spxentry{fredpy.series.monthtoannual()}\spxextra{built-in function}}

\begin{fulllineitems}
\phantomsection\label{\detokenize{series_class:fredpy.series.monthtoannual}}\pysiglinewithargsret{\sphinxbfcode{\sphinxupquote{monthtoannual}}}{\emph{method='average'}}{}
Converts monthly data to annual data.
\begin{quote}\begin{description}
\item[{Parameters}] \leavevmode
\sphinxstyleliteralstrong{\sphinxupquote{method}} (\sphinxstyleliteralemphasis{\sphinxupquote{string}}) \textendash{} If ‘average’, use the average values over each twelve month interval (default), if ‘sum,’ use the sum of the values over each twelve month interval, and if ‘end’ use the values at the end of each twelve month interval.

\item[{Returns}] \leavevmode
{\hyperref[\detokenize{series_class:fredpy.series}]{\sphinxcrossref{\sphinxcode{\sphinxupquote{fredpy.series}}}}}

\end{description}\end{quote}

\end{fulllineitems}

\index{fredpy.series.monthtoquarter() (built-in function)@\spxentry{fredpy.series.monthtoquarter()}\spxextra{built-in function}}

\begin{fulllineitems}
\phantomsection\label{\detokenize{series_class:fredpy.series.monthtoquarter}}\pysiglinewithargsret{\sphinxbfcode{\sphinxupquote{monthtoquarter}}}{\emph{method='average'}}{}
Converts monthly data to quarterly data.
\begin{quote}\begin{description}
\item[{Parameters}] \leavevmode
\sphinxstyleliteralstrong{\sphinxupquote{method}} (\sphinxstyleliteralemphasis{\sphinxupquote{string}}) \textendash{} If ‘average’, use the average values over each three month interval (default), if ‘sum,’ use the sum of the values over each three month interval, and if ‘end’ use the values at the end of each three month interval.

\item[{Returns}] \leavevmode
{\hyperref[\detokenize{series_class:fredpy.series}]{\sphinxcrossref{\sphinxcode{\sphinxupquote{fredpy.series}}}}}

\end{description}\end{quote}

\end{fulllineitems}

\index{fredpy.series.pc() (built-in function)@\spxentry{fredpy.series.pc()}\spxextra{built-in function}}

\begin{fulllineitems}
\phantomsection\label{\detokenize{series_class:fredpy.series.pc}}\pysiglinewithargsret{\sphinxbfcode{\sphinxupquote{pc}}}{\emph{log=True}, \emph{method='backward'}, \emph{annualized=False}}{}
Computes the percentage change in the data from the preceding period.
\begin{quote}\begin{description}
\item[{Parameters}] \leavevmode\begin{itemize}
\item {} 
\sphinxstyleliteralstrong{\sphinxupquote{log}} (\sphinxhref{https://docs.python.org/3/library/functions.html\#bool}{\sphinxstyleliteralemphasis{\sphinxupquote{bool}}}) \textendash{} If True, computes the percentage change as \(100\cdot\log(x_{t}/x_{t-1})\). If False, compute the percentage change as \(100\cdot\left( x_{t}/x_{t-1} - 1\right)\).

\item {} 
\sphinxstyleliteralstrong{\sphinxupquote{method}} (\sphinxstyleliteralemphasis{\sphinxupquote{string}}) \textendash{} If ‘backward’, compute percentage change from the previous period. If ‘forward’, compute percentage change from current to subsequent period.

\item {} 
\sphinxstyleliteralstrong{\sphinxupquote{annualized}} (\sphinxhref{https://docs.python.org/3/library/functions.html\#bool}{\sphinxstyleliteralemphasis{\sphinxupquote{bool}}}) \textendash{} If True, percentage change is annualized by multipying the simple percentage change by the number of data observations per year. E.g., if the data are monthly, then the annualized percentage change is \(4\cdot 100\cdot\log(x_{t}/x_{t-1})\).

\end{itemize}

\item[{Returns}] \leavevmode
{\hyperref[\detokenize{series_class:fredpy.series}]{\sphinxcrossref{\sphinxcode{\sphinxupquote{fredpy.series}}}}}

\end{description}\end{quote}

\end{fulllineitems}

\index{fredpy.series.percapita() (built-in function)@\spxentry{fredpy.series.percapita()}\spxextra{built-in function}}

\begin{fulllineitems}
\phantomsection\label{\detokenize{series_class:fredpy.series.percapita}}\pysiglinewithargsret{\sphinxbfcode{\sphinxupquote{percapita}}}{\emph{total\_pop=True}}{}
Transforms the data into per capita terms (US) by dividing by one of two measures of the total population.
\begin{quote}\begin{description}
\item[{Parameters}] \leavevmode
\sphinxstyleliteralstrong{\sphinxupquote{total\_pop}} (\sphinxstyleliteralemphasis{\sphinxupquote{string}}) \textendash{} If \sphinxcode{\sphinxupquote{total\_pop == True}}, then use the toal population (Default). Else, use Civilian noninstitutional population defined as persons 16 years of age and older.

\item[{Returns}] \leavevmode
{\hyperref[\detokenize{series_class:fredpy.series}]{\sphinxcrossref{\sphinxcode{\sphinxupquote{fredpy.series}}}}}

\end{description}\end{quote}

\end{fulllineitems}

\index{fredpy.series.plus() (built-in function)@\spxentry{fredpy.series.plus()}\spxextra{built-in function}}

\begin{fulllineitems}
\phantomsection\label{\detokenize{series_class:fredpy.series.plus}}\pysiglinewithargsret{\sphinxbfcode{\sphinxupquote{plus}}}{\emph{series2}}{}
Adds the data from the current fredpy series to the data from \sphinxcode{\sphinxupquote{series2}}.
\begin{quote}\begin{description}
\item[{Parameters}] \leavevmode
\sphinxstyleliteralstrong{\sphinxupquote{series2}} ({\hyperref[\detokenize{series_class:fredpy.series}]{\sphinxcrossref{\sphinxstyleliteralemphasis{\sphinxupquote{fredpy.series}}}}}) \textendash{} A :py:class:\sphinxcode{\sphinxupquote{fredpy.series}} instance.

\item[{Returns}] \leavevmode
{\hyperref[\detokenize{series_class:fredpy.series}]{\sphinxcrossref{\sphinxcode{\sphinxupquote{fredpy.series}}}}}

\end{description}\end{quote}

\end{fulllineitems}

\index{fredpy.series.quartertoannual() (built-in function)@\spxentry{fredpy.series.quartertoannual()}\spxextra{built-in function}}

\begin{fulllineitems}
\phantomsection\label{\detokenize{series_class:fredpy.series.quartertoannual}}\pysiglinewithargsret{\sphinxbfcode{\sphinxupquote{quartertoannual}}}{\emph{method='average'}}{}
Converts quarterly data to annual data.
\begin{quote}\begin{description}
\item[{Parameters}] \leavevmode
\sphinxstyleliteralstrong{\sphinxupquote{method}} (\sphinxstyleliteralemphasis{\sphinxupquote{string}}) \textendash{} If ‘average’, use the average values over each four quarter interval (default), if ‘sum,’ use the sum of the values over each four quarter interval, and if ‘end’ use the values at the end of each four quarter interval.

\item[{Returns}] \leavevmode
{\hyperref[\detokenize{series_class:fredpy.series}]{\sphinxcrossref{\sphinxcode{\sphinxupquote{fredpy.series}}}}}

\end{description}\end{quote}

\end{fulllineitems}

\index{fredpy.series.recent() (built-in function)@\spxentry{fredpy.series.recent()}\spxextra{built-in function}}

\begin{fulllineitems}
\phantomsection\label{\detokenize{series_class:fredpy.series.recent}}\pysiglinewithargsret{\sphinxbfcode{\sphinxupquote{recent}}}{\emph{N}}{}
Restrict the data to the most recent N observations.
\begin{quote}\begin{description}
\item[{Parameters}] \leavevmode
\sphinxstyleliteralstrong{\sphinxupquote{N}} (\sphinxhref{https://docs.python.org/3/library/functions.html\#int}{\sphinxstyleliteralemphasis{\sphinxupquote{int}}}) \textendash{} Number of periods to include in the data window.

\item[{Returns}] \leavevmode
{\hyperref[\detokenize{series_class:fredpy.series}]{\sphinxcrossref{\sphinxcode{\sphinxupquote{fredpy.series}}}}}

\end{description}\end{quote}

\end{fulllineitems}

\index{fredpy.series.recessions() (built-in function)@\spxentry{fredpy.series.recessions()}\spxextra{built-in function}}

\begin{fulllineitems}
\phantomsection\label{\detokenize{series_class:fredpy.series.recessions}}\pysiglinewithargsret{\sphinxbfcode{\sphinxupquote{recessions}}}{\emph{color='0.5'}, \emph{alpha = 0.5}}{}
Creates recession bars for plots. Should be used after a plot has been made but before either (1) a new plot is created or (2) a show command is issued.
\begin{quote}\begin{description}
\item[{Parameters}] \leavevmode\begin{itemize}
\item {} 
\sphinxstyleliteralstrong{\sphinxupquote{color}} (\sphinxstyleliteralemphasis{\sphinxupquote{string}}) \textendash{} Color of the bars. Default: ‘0.5’.

\item {} 
\sphinxstyleliteralstrong{\sphinxupquote{alpha}} (\sphinxhref{https://docs.python.org/3/library/functions.html\#float}{\sphinxstyleliteralemphasis{\sphinxupquote{float}}}) \textendash{} Transparency of the recession bars. Must be between 0 and 1. Default: 0.5.

\end{itemize}

\item[{Returns}] \leavevmode


\end{description}\end{quote}

\end{fulllineitems}

\index{fredpy.series.times() (built-in function)@\spxentry{fredpy.series.times()}\spxextra{built-in function}}

\begin{fulllineitems}
\phantomsection\label{\detokenize{series_class:fredpy.series.times}}\pysiglinewithargsret{\sphinxbfcode{\sphinxupquote{times}}}{\emph{series2}}{}
Multiplies the data from the current fredpy series with the data from \sphinxcode{\sphinxupquote{series2}}.
\begin{quote}\begin{description}
\item[{Parameters}] \leavevmode
\sphinxstyleliteralstrong{\sphinxupquote{series2}} ({\hyperref[\detokenize{series_class:fredpy.series}]{\sphinxcrossref{\sphinxstyleliteralemphasis{\sphinxupquote{fredpy.series}}}}}) \textendash{} A {\hyperref[\detokenize{series_class:fredpy.series}]{\sphinxcrossref{\sphinxcode{\sphinxupquote{fredpy.series}}}}} instance.

\item[{Returns}] \leavevmode
{\hyperref[\detokenize{series_class:fredpy.series}]{\sphinxcrossref{\sphinxcode{\sphinxupquote{fredpy.series}}}}}

\end{description}\end{quote}

\end{fulllineitems}

\index{fredpy.series.window() (built-in function)@\spxentry{fredpy.series.window()}\spxextra{built-in function}}

\begin{fulllineitems}
\phantomsection\label{\detokenize{series_class:fredpy.series.window}}\pysiglinewithargsret{\sphinxbfcode{\sphinxupquote{window}}}{\emph{win}}{}
Restricts the data to the most recent N observations.
\begin{quote}\begin{description}
\item[{Parameters}] \leavevmode
\sphinxstyleliteralstrong{\sphinxupquote{win}} (\sphinxhref{https://docs.python.org/3/library/stdtypes.html\#list}{\sphinxstyleliteralemphasis{\sphinxupquote{list}}}) \textendash{} is an ordered pair: \sphinxcode{\sphinxupquote{win = {[}win\_min, win\_max{]}}} where \sphinxcode{\sphinxupquote{win\_min}} is the date of the minimum date desired and \sphinxcode{\sphinxupquote{win\_max}} is the date of the maximum date. Date strings must be entered in either YYYY-MM-DD or MM-DD-YYYY format.

\item[{Returns}] \leavevmode
{\hyperref[\detokenize{series_class:fredpy.series}]{\sphinxcrossref{\sphinxcode{\sphinxupquote{fredpy.series}}}}}

\end{description}\end{quote}

\end{fulllineitems}

\end{quote}

\end{fulllineitems}



\section{Additional \sphinxstyleliteralintitle{\sphinxupquote{fredpy}} Functions}
\label{\detokenize{additional_functions:additional-fredpy-functions}}\label{\detokenize{additional_functions::doc}}\index{fredpy.date\_times() (built-in function)@\spxentry{fredpy.date\_times()}\spxextra{built-in function}}

\begin{fulllineitems}
\phantomsection\label{\detokenize{additional_functions:fredpy.date_times}}\pysiglinewithargsret{\sphinxcode{\sphinxupquote{fredpy.}}\sphinxbfcode{\sphinxupquote{date\_times}}}{\emph{date\_strings}}{}
Converts a list of date strings in yyyy-mm-dd format to datetime.
\begin{quote}\begin{description}
\item[{Parameters}] \leavevmode
\sphinxstyleliteralstrong{\sphinxupquote{date\_strings}} (\sphinxhref{https://docs.python.org/3/library/stdtypes.html\#list}{\sphinxstyleliteralemphasis{\sphinxupquote{list}}}) \textendash{} a list of date strings formated as: ‘yyyy-mm-dd’.

\item[{Returns}] \leavevmode
\sphinxcode{\sphinxupquote{numpy.ndarray}}

\end{description}\end{quote}

\end{fulllineitems}

\index{fredpy.divide() (built-in function)@\spxentry{fredpy.divide()}\spxextra{built-in function}}

\begin{fulllineitems}
\phantomsection\label{\detokenize{additional_functions:fredpy.divide}}\pysiglinewithargsret{\sphinxcode{\sphinxupquote{fredpy.}}\sphinxbfcode{\sphinxupquote{divide}}}{\emph{series1}, \emph{series2}}{}
Divides the data from \sphinxcode{\sphinxupquote{series1}} by the data from \sphinxcode{\sphinxupquote{series2}}.
\begin{quote}\begin{description}
\item[{Parameters}] \leavevmode\begin{itemize}
\item {} 
\sphinxstyleliteralstrong{\sphinxupquote{series1}} ({\hyperref[\detokenize{series_class:fredpy.series}]{\sphinxcrossref{\sphinxstyleliteralemphasis{\sphinxupquote{fredpy.series}}}}}) \textendash{} A {\hyperref[\detokenize{series_class:fredpy.series}]{\sphinxcrossref{\sphinxcode{\sphinxupquote{fredpy.series}}}}} object.

\item {} 
\sphinxstyleliteralstrong{\sphinxupquote{series2}} ({\hyperref[\detokenize{series_class:fredpy.series}]{\sphinxcrossref{\sphinxstyleliteralemphasis{\sphinxupquote{fredpy.series}}}}}) \textendash{} A {\hyperref[\detokenize{series_class:fredpy.series}]{\sphinxcrossref{\sphinxcode{\sphinxupquote{fredpy.series}}}}} object.

\end{itemize}

\item[{Returns}] \leavevmode
{\hyperref[\detokenize{series_class:fredpy.series}]{\sphinxcrossref{\sphinxcode{\sphinxupquote{fredpy.series}}}}}

\end{description}\end{quote}

\end{fulllineitems}

\index{fredpy.plus() (built-in function)@\spxentry{fredpy.plus()}\spxextra{built-in function}}

\begin{fulllineitems}
\phantomsection\label{\detokenize{additional_functions:fredpy.plus}}\pysiglinewithargsret{\sphinxcode{\sphinxupquote{fredpy.}}\sphinxbfcode{\sphinxupquote{plus}}}{\emph{series1}, \emph{series2}}{}
Adds the data from \sphinxcode{\sphinxupquote{series1}} to the data from \sphinxcode{\sphinxupquote{series2}}.
\begin{quote}\begin{description}
\item[{Parameters}] \leavevmode\begin{itemize}
\item {} 
\sphinxstyleliteralstrong{\sphinxupquote{series1}} ({\hyperref[\detokenize{series_class:fredpy.series}]{\sphinxcrossref{\sphinxstyleliteralemphasis{\sphinxupquote{fredpy.series}}}}}) \textendash{} A {\hyperref[\detokenize{series_class:fredpy.series}]{\sphinxcrossref{\sphinxcode{\sphinxupquote{fredpy.series}}}}} object.

\item {} 
\sphinxstyleliteralstrong{\sphinxupquote{series2}} ({\hyperref[\detokenize{series_class:fredpy.series}]{\sphinxcrossref{\sphinxstyleliteralemphasis{\sphinxupquote{fredpy.series}}}}}) \textendash{} A {\hyperref[\detokenize{series_class:fredpy.series}]{\sphinxcrossref{\sphinxcode{\sphinxupquote{fredpy.series}}}}} object.

\end{itemize}

\item[{Returns}] \leavevmode
{\hyperref[\detokenize{series_class:fredpy.series}]{\sphinxcrossref{\sphinxcode{\sphinxupquote{fredpy.series}}}}}

\end{description}\end{quote}

\end{fulllineitems}

\index{fredpy.quickplot() (built-in function)@\spxentry{fredpy.quickplot()}\spxextra{built-in function}}

\begin{fulllineitems}
\phantomsection\label{\detokenize{additional_functions:fredpy.quickplot}}\pysiglinewithargsret{\sphinxcode{\sphinxupquote{fredpy.}}\sphinxbfcode{\sphinxupquote{quickplot}}}{\emph{fred\_series}, \emph{year\_mult=10}, \emph{show=True}, \emph{recess=False}, \emph{style='default'}, \emph{save=False}, \emph{filename='file'}, \emph{linewidth=2}, \emph{alpha = 0.7}}{}
Create a plot of a FRED data series.
\begin{quote}\begin{description}
\item[{Parameters}] \leavevmode\begin{itemize}
\item {} 
\sphinxstyleliteralstrong{\sphinxupquote{fred\_series}} ({\hyperref[\detokenize{series_class:fredpy.series}]{\sphinxcrossref{\sphinxstyleliteralemphasis{\sphinxupquote{fredpy.series}}}}}) \textendash{} A {\hyperref[\detokenize{series_class:fredpy.series}]{\sphinxcrossref{\sphinxcode{\sphinxupquote{fredpy.series}}}}} object.

\item {} 
\sphinxstyleliteralstrong{\sphinxupquote{year\_mult}} (\sphinxhref{https://docs.python.org/3/library/functions.html\#int}{\sphinxstyleliteralemphasis{\sphinxupquote{int}}}) \textendash{} Interval between year ticks on the x-axis. Default: 10.

\item {} 
\sphinxstyleliteralstrong{\sphinxupquote{show}} (\sphinxhref{https://docs.python.org/3/library/functions.html\#bool}{\sphinxstyleliteralemphasis{\sphinxupquote{bool}}}) \textendash{} Show the plot? Default: True.

\item {} 
\sphinxstyleliteralstrong{\sphinxupquote{recess}} (\sphinxhref{https://docs.python.org/3/library/functions.html\#bool}{\sphinxstyleliteralemphasis{\sphinxupquote{bool}}}) \textendash{} Show recession bars in plot? Default: False.

\item {} 
\sphinxstyleliteralstrong{\sphinxupquote{style}} (\sphinxhref{https://docs.python.org/3/library/functions.html\#bool}{\sphinxstyleliteralemphasis{\sphinxupquote{bool}}}) \textendash{} Matplotlib style. Default: ‘default’.

\item {} 
\sphinxstyleliteralstrong{\sphinxupquote{save}} (\sphinxhref{https://docs.python.org/3/library/functions.html\#bool}{\sphinxstyleliteralemphasis{\sphinxupquote{bool}}}) \textendash{} Save the image to file? Default: False.

\item {} 
\sphinxstyleliteralstrong{\sphinxupquote{filename}} (\sphinxstyleliteralemphasis{\sphinxupquote{string}}) \textendash{} Name of file to which image is saved \sphinxstyleemphasis{without an extension}. Default: \sphinxcode{\sphinxupquote{'file'}}.

\item {} 
\sphinxstyleliteralstrong{\sphinxupquote{linewidth}} (\sphinxhref{https://docs.python.org/3/library/functions.html\#float}{\sphinxstyleliteralemphasis{\sphinxupquote{float}}}) \textendash{} Width of plotted line. Default: 2.

\item {} 
\sphinxstyleliteralstrong{\sphinxupquote{alpha}} (\sphinxhref{https://docs.python.org/3/library/functions.html\#float}{\sphinxstyleliteralemphasis{\sphinxupquote{float}}}) \textendash{} Transparency of the recession bars. Must be between 0 and 1. Default: 0.7.

\end{itemize}

\item[{Returns}] \leavevmode


\end{description}\end{quote}

\end{fulllineitems}

\index{fredpy.minus() (built-in function)@\spxentry{fredpy.minus()}\spxextra{built-in function}}

\begin{fulllineitems}
\phantomsection\label{\detokenize{additional_functions:fredpy.minus}}\pysiglinewithargsret{\sphinxcode{\sphinxupquote{fredpy.}}\sphinxbfcode{\sphinxupquote{minus}}}{\emph{series1}, \emph{series2}}{}
Subtracts the data from \sphinxcode{\sphinxupquote{series2}} from the data from \sphinxcode{\sphinxupquote{series1}}.
\begin{quote}\begin{description}
\item[{Parameters}] \leavevmode\begin{itemize}
\item {} 
\sphinxstyleliteralstrong{\sphinxupquote{series1}} ({\hyperref[\detokenize{series_class:fredpy.series}]{\sphinxcrossref{\sphinxstyleliteralemphasis{\sphinxupquote{fredpy.series}}}}}) \textendash{} A {\hyperref[\detokenize{series_class:fredpy.series}]{\sphinxcrossref{\sphinxcode{\sphinxupquote{fredpy.series}}}}} object.

\item {} 
\sphinxstyleliteralstrong{\sphinxupquote{series2}} ({\hyperref[\detokenize{series_class:fredpy.series}]{\sphinxcrossref{\sphinxstyleliteralemphasis{\sphinxupquote{fredpy.series}}}}}) \textendash{} A {\hyperref[\detokenize{series_class:fredpy.series}]{\sphinxcrossref{\sphinxcode{\sphinxupquote{fredpy.series}}}}} object.

\end{itemize}

\item[{Returns}] \leavevmode
{\hyperref[\detokenize{series_class:fredpy.series}]{\sphinxcrossref{\sphinxcode{\sphinxupquote{fredpy.series}}}}}

\end{description}\end{quote}

\end{fulllineitems}

\index{fredpy.times() (built-in function)@\spxentry{fredpy.times()}\spxextra{built-in function}}

\begin{fulllineitems}
\phantomsection\label{\detokenize{additional_functions:fredpy.times}}\pysiglinewithargsret{\sphinxcode{\sphinxupquote{fredpy.}}\sphinxbfcode{\sphinxupquote{times}}}{\emph{series1}, \emph{series2}}{}
Multiplies the data from \sphinxcode{\sphinxupquote{series1}} with the data from \sphinxcode{\sphinxupquote{series2}}.
\begin{quote}\begin{description}
\item[{Parameters}] \leavevmode\begin{itemize}
\item {} 
\sphinxstyleliteralstrong{\sphinxupquote{series1}} ({\hyperref[\detokenize{series_class:fredpy.series}]{\sphinxcrossref{\sphinxstyleliteralemphasis{\sphinxupquote{fredpy.series}}}}}) \textendash{} A {\hyperref[\detokenize{series_class:fredpy.series}]{\sphinxcrossref{\sphinxcode{\sphinxupquote{fredpy.series}}}}} object.

\item {} 
\sphinxstyleliteralstrong{\sphinxupquote{series2}} ({\hyperref[\detokenize{series_class:fredpy.series}]{\sphinxcrossref{\sphinxstyleliteralemphasis{\sphinxupquote{fredpy.series}}}}}) \textendash{} A {\hyperref[\detokenize{series_class:fredpy.series}]{\sphinxcrossref{\sphinxcode{\sphinxupquote{fredpy.series}}}}} object.

\end{itemize}

\item[{Returns}] \leavevmode
{\hyperref[\detokenize{series_class:fredpy.series}]{\sphinxcrossref{\sphinxcode{\sphinxupquote{fredpy.series}}}}}

\end{description}\end{quote}

\end{fulllineitems}

\index{fredpy.toFredSeries() (built-in function)@\spxentry{fredpy.toFredSeries()}\spxextra{built-in function}}

\begin{fulllineitems}
\phantomsection\label{\detokenize{additional_functions:fredpy.toFredSeries}}\pysiglinewithargsret{\sphinxcode{\sphinxupquote{fredpy.}}\sphinxbfcode{\sphinxupquote{toFredSeries}}}{\emph{data}, \emph{dates}, \emph{frequency=''}, \emph{frequency\_short=''}, \emph{last\_updated=''}, \emph{notes=''}, \emph{release=''}, \emph{seasonal\_adjustment=''}, \emph{seasonal\_adjustment\_short=''}, \emph{series\_id=''}, \emph{source=''}, \emph{t=0}, \emph{title=''}, \emph{units=''}, \emph{units\_short=''}}{}
Create a {\hyperref[\detokenize{series_class:fredpy.series}]{\sphinxcrossref{\sphinxcode{\sphinxupquote{fredpy.series}}}}} from time series data not obtained from FRED.
\begin{quote}\begin{description}
\item[{Parameters}] \leavevmode\begin{itemize}
\item {} 
\sphinxstyleliteralstrong{\sphinxupquote{data}} (\sphinxstyleliteralemphasis{\sphinxupquote{numpy.ndarray}}\sphinxstyleliteralemphasis{\sphinxupquote{, }}\sphinxstyleliteralemphasis{\sphinxupquote{Pandas.Series}}\sphinxstyleliteralemphasis{\sphinxupquote{, or }}\sphinxhref{https://docs.python.org/3/library/stdtypes.html\#list}{\sphinxstyleliteralemphasis{\sphinxupquote{list}}}) \textendash{} Data values.

\item {} 
\sphinxstyleliteralstrong{\sphinxupquote{dates}} (\sphinxhref{https://docs.python.org/3/library/stdtypes.html\#list}{\sphinxstyleliteralemphasis{\sphinxupquote{list}}}\sphinxstyleliteralemphasis{\sphinxupquote{ or }}\sphinxstyleliteralemphasis{\sphinxupquote{numpy.ndarry}}) \textendash{} Array or list of dates. Elements formatted as either string (YYYY-MM-DD or MM-DD-YYYY) or \sphinxcode{\sphinxupquote{pandas.tslib.Timestamp}}.

\item {} 
\sphinxstyleliteralstrong{\sphinxupquote{frequency}} (\sphinxhref{https://docs.python.org/3/library/stdtypes.html\#str}{\sphinxstyleliteralemphasis{\sphinxupquote{str}}}) \textendash{} Observation frequency. Options: ‘’, ‘Daily’, ‘Weekly’, ‘Monthly’, ‘Quarterly’, or ‘Annual’. Default: empty string.

\item {} 
\sphinxstyleliteralstrong{\sphinxupquote{frequency\_short}} (\sphinxhref{https://docs.python.org/3/library/stdtypes.html\#str}{\sphinxstyleliteralemphasis{\sphinxupquote{str}}}) \textendash{} Observation frequency abbreviated. Options: ‘’, ‘D’, ‘W’, ‘M’, ‘Q’, or ‘A’. Default: empty string.

\item {} 
\sphinxstyleliteralstrong{\sphinxupquote{last\_updated}} (\sphinxhref{https://docs.python.org/3/library/stdtypes.html\#str}{\sphinxstyleliteralemphasis{\sphinxupquote{str}}}) \textendash{} Date data was last updated. Default: empty string.

\item {} 
\sphinxstyleliteralstrong{\sphinxupquote{notes}} (\sphinxhref{https://docs.python.org/3/library/stdtypes.html\#str}{\sphinxstyleliteralemphasis{\sphinxupquote{str}}}) \textendash{} Default: empty string.

\item {} 
\sphinxstyleliteralstrong{\sphinxupquote{release}} (\sphinxhref{https://docs.python.org/3/library/stdtypes.html\#str}{\sphinxstyleliteralemphasis{\sphinxupquote{str}}}) \textendash{} Notes about data. Default: empty string.

\item {} 
\sphinxstyleliteralstrong{\sphinxupquote{seasonal\_adjustment}} (\sphinxhref{https://docs.python.org/3/library/stdtypes.html\#str}{\sphinxstyleliteralemphasis{\sphinxupquote{str}}}) \textendash{} Default: empty string.

\item {} 
\sphinxstyleliteralstrong{\sphinxupquote{seasonal\_adjustment\_short}} (\sphinxhref{https://docs.python.org/3/library/stdtypes.html\#str}{\sphinxstyleliteralemphasis{\sphinxupquote{str}}}) \textendash{} Default: empty string.

\item {} 
\sphinxstyleliteralstrong{\sphinxupquote{series\_id}} (\sphinxhref{https://docs.python.org/3/library/stdtypes.html\#str}{\sphinxstyleliteralemphasis{\sphinxupquote{str}}}) \textendash{} FRED series ID. Default: empty string.

\item {} 
\sphinxstyleliteralstrong{\sphinxupquote{source}} (\sphinxhref{https://docs.python.org/3/library/stdtypes.html\#str}{\sphinxstyleliteralemphasis{\sphinxupquote{str}}}) \textendash{} Source of the data. Default: empty string.

\item {} 
\sphinxstyleliteralstrong{\sphinxupquote{t}} (\sphinxhref{https://docs.python.org/3/library/functions.html\#int}{\sphinxstyleliteralemphasis{\sphinxupquote{int}}}) \textendash{} Number of observations per year. Default: 0

\item {} 
\sphinxstyleliteralstrong{\sphinxupquote{title}} (\sphinxhref{https://docs.python.org/3/library/stdtypes.html\#str}{\sphinxstyleliteralemphasis{\sphinxupquote{str}}}) \textendash{} Title of the data. Default: empty string.

\item {} 
\sphinxstyleliteralstrong{\sphinxupquote{units}} (\sphinxhref{https://docs.python.org/3/library/stdtypes.html\#str}{\sphinxstyleliteralemphasis{\sphinxupquote{str}}}) \textendash{} Units of the data. Default: empty string.

\item {} 
\sphinxstyleliteralstrong{\sphinxupquote{units\_short}} (\sphinxhref{https://docs.python.org/3/library/stdtypes.html\#str}{\sphinxstyleliteralemphasis{\sphinxupquote{str}}}) \textendash{} Units of the data. Abbreviated. Default: empty string.

\end{itemize}

\item[{Returns}] \leavevmode
{\hyperref[\detokenize{series_class:fredpy.series}]{\sphinxcrossref{\sphinxcode{\sphinxupquote{fredpy.series}}}}}

\end{description}\end{quote}

\end{fulllineitems}

\index{fredpy.window\_equalize() (built-in function)@\spxentry{fredpy.window\_equalize()}\spxextra{built-in function}}

\begin{fulllineitems}
\phantomsection\label{\detokenize{additional_functions:fredpy.window_equalize}}\pysiglinewithargsret{\sphinxcode{\sphinxupquote{fredpy.}}\sphinxbfcode{\sphinxupquote{window\_equalize}}}{\emph{series\_list}}{}
Adjusts the date windows for a collection of fredpy.series objects to the smallest common window.
\begin{quote}\begin{description}
\item[{Parameters}] \leavevmode
\sphinxstyleliteralstrong{\sphinxupquote{series\_list}} (\sphinxhref{https://docs.python.org/3/library/stdtypes.html\#list}{\sphinxstyleliteralemphasis{\sphinxupquote{list}}}) \textendash{} A list of {\hyperref[\detokenize{series_class:fredpy.series}]{\sphinxcrossref{\sphinxcode{\sphinxupquote{fredpy.series}}}}} objects

\item[{Returns}] \leavevmode


\end{description}\end{quote}

\end{fulllineitems}



\section{\sphinxstyleliteralintitle{\sphinxupquote{fredpy}} Examples}
\label{\detokenize{fredpy_examples:fredpy-Examples}}\label{\detokenize{fredpy_examples::doc}}
\fvset{hllines={, ,}}%
\begin{Verbatim}[commandchars=\\\{\}]
\textcolor{nbsphinxin}{In [1]: }\PYG{k+kn}{import} \PYG{n+nn}{pandas} \PYG{k+kn}{as} \PYG{n+nn}{pd}
        \PYG{k+kn}{import} \PYG{n+nn}{numpy} \PYG{k+kn}{as} \PYG{n+nn}{np}
        \PYG{k+kn}{import} \PYG{n+nn}{fredpy} \PYG{k+kn}{as} \PYG{n+nn}{fp}
        \PYG{k+kn}{import} \PYG{n+nn}{matplotlib.pyplot} \PYG{k+kn}{as} \PYG{n+nn}{plt}
        \PYG{n}{plt}\PYG{o}{.}\PYG{n}{style}\PYG{o}{.}\PYG{n}{use}\PYG{p}{(}\PYG{l+s+s1}{\PYGZsq{}}\PYG{l+s+s1}{classic}\PYG{l+s+s1}{\PYGZsq{}}\PYG{p}{)}
        \PYG{o}{\PYGZpc{}}\PYG{k}{matplotlib} inline
\end{Verbatim}


\subsection{Load API key}
\label{\detokenize{fredpy_examples:Load-API-key}}
First apply for an API key for FRED here:
\sphinxurl{https://research.stlouisfed.org/docs/api/api\_key.html}. The API key is a
32 character string that is required for making requests from FRED. Save
your API key in the \sphinxcode{\sphinxupquote{fp}} namespace by either setting the
\sphinxcode{\sphinxupquote{fp.api\_key}} directly:

\fvset{hllines={, ,}}%
\begin{Verbatim}[commandchars=\\\{\}]
\textcolor{nbsphinxin}{In [2]: }\PYG{n}{fp}\PYG{o}{.}\PYG{n}{api\PYGZus{}key} \PYG{o}{=} \PYG{l+s+s1}{\PYGZsq{}}\PYG{l+s+s1}{\PYGZsh{}\PYGZsh{}\PYGZsh{}\PYGZsh{}\PYGZsh{}\PYGZsh{}\PYGZsh{}\PYGZsh{}\PYGZsh{}\PYGZsh{}\PYGZsh{}\PYGZsh{}\PYGZsh{}\PYGZsh{}\PYGZsh{}\PYGZsh{}\PYGZsh{}\PYGZsh{}\PYGZsh{}\PYGZsh{}\PYGZsh{}\PYGZsh{}\PYGZsh{}\PYGZsh{}\PYGZsh{}\PYGZsh{}\PYGZsh{}\PYGZsh{}\PYGZsh{}\PYGZsh{}\PYGZsh{}\PYGZsh{}}\PYG{l+s+s1}{\PYGZsq{}}
\end{Verbatim}

or by reading from a text file containing only the text of the API key
in the first line:

\fvset{hllines={, ,}}%
\begin{Verbatim}[commandchars=\\\{\}]
\textcolor{nbsphinxin}{In [3]: }\PYG{n}{fp}\PYG{o}{.}\PYG{n}{api\PYGZus{}key} \PYG{o}{=} \PYG{n}{fp}\PYG{o}{.}\PYG{n}{load\PYGZus{}api\PYGZus{}key}\PYG{p}{(}\PYG{l+s+s1}{\PYGZsq{}}\PYG{l+s+s1}{fred\PYGZus{}api\PYGZus{}key.txt}\PYG{l+s+s1}{\PYGZsq{}}\PYG{p}{)}
\end{Verbatim}

If \sphinxcode{\sphinxupquote{fred\_api\_key.txt}} is not in the same directory as your program
file, then you must supply the full path of the file.


\subsection{Preliminary example}
\label{\detokenize{fredpy_examples:Preliminary-example}}
Downloading and plotting unemployment rate data for the US is easy with
\sphinxcode{\sphinxupquote{fredpy}}:

\fvset{hllines={, ,}}%
\begin{Verbatim}[commandchars=\\\{\}]
\textcolor{nbsphinxin}{In [4]: }\PYG{n}{u} \PYG{o}{=} \PYG{n}{fp}\PYG{o}{.}\PYG{n}{series}\PYG{p}{(}\PYG{l+s+s1}{\PYGZsq{}}\PYG{l+s+s1}{UNRATE}\PYG{l+s+s1}{\PYGZsq{}}\PYG{p}{)}
        \PYG{n}{plt}\PYG{o}{.}\PYG{n}{plot\PYGZus{}date}\PYG{p}{(}\PYG{n}{u}\PYG{o}{.}\PYG{n}{datetimes}\PYG{p}{,}\PYG{n}{u}\PYG{o}{.}\PYG{n}{data}\PYG{p}{,}\PYG{l+s+s1}{\PYGZsq{}}\PYG{l+s+s1}{\PYGZhy{}}\PYG{l+s+s1}{\PYGZsq{}}\PYG{p}{,}\PYG{n}{lw}\PYG{o}{=}\PYG{l+m+mi}{3}\PYG{p}{,}\PYG{n}{alpha} \PYG{o}{=} \PYG{l+m+mf}{0.65}\PYG{p}{)}
        \PYG{n}{plt}\PYG{o}{.}\PYG{n}{grid}\PYG{p}{(}\PYG{p}{)}
\end{Verbatim}

\noindent\sphinxincludegraphics{{fredpy_examples_8_0}.png}


\subsection{A closer look at \sphinxstyleliteralintitle{\sphinxupquote{fredpy}} using real GDP data}
\label{\detokenize{fredpy_examples:A-closer-look-at-fredpy-using-real-GDP-data}}
Use \sphinxcode{\sphinxupquote{fredpy}} to download real GDP data. The FRED page for real GDP: \sphinxurl{https://fred.stlouisfed.org/series/GDPC1}. Note that the series ID - \sphinxcode{\sphinxupquote{GDPC1}} - is in the URL and is visible in several places on the page.

The data in text format is located at: \sphinxurl{https://fred.stlouisfed.org/data/gdpc1.txt}. When supplied with the series ID \sphinxcode{\sphinxupquote{GDPC1}}, \sphinxcode{\sphinxupquote{fredpy}} visits the the URL for the text-formatted data, reads the information on the page, and stores the data as attributes of a {\hyperref[\detokenize{series_class:fredpy.series}]{\sphinxcrossref{\sphinxcode{\sphinxupquote{fredpy.series}}}}} instance.

\fvset{hllines={, ,}}%
\begin{Verbatim}[commandchars=\\\{\}]
\textcolor{nbsphinxin}{In [5]: }\PYG{c+c1}{\PYGZsh{} Download quarterly real GDP data using {}`fredpy{}`. Save the data in a variable called gdp}
        \PYG{n}{gdp} \PYG{o}{=} \PYG{n}{fp}\PYG{o}{.}\PYG{n}{series}\PYG{p}{(}\PYG{l+s+s1}{\PYGZsq{}}\PYG{l+s+s1}{gdpc1}\PYG{l+s+s1}{\PYGZsq{}}\PYG{p}{)}
        
        \PYG{c+c1}{\PYGZsh{} Note that gdp is an instance of the {}`fredpy.series{}` class}
        \PYG{k}{print}\PYG{p}{(}\PYG{n+nb}{type}\PYG{p}{(}\PYG{n}{gdp}\PYG{p}{)}\PYG{p}{)}
\end{Verbatim}



%
\begin{OriginalVerbatim}[commandchars=\\\{\}]
<class 'fredpy.series'>
\end{OriginalVerbatim}
% The following \relax is needed to avoid problems with adjacent ANSI
% cells and some other stuff (e.g. bullet lists) following ANSI cells.
% See https://github.com/sphinx-doc/sphinx/issues/3594
\relax


\subsubsection{Attributes}
\label{\detokenize{fredpy_examples:Attributes}}
A {\hyperref[\detokenize{series_class:fredpy.series}]{\sphinxcrossref{\sphinxcode{\sphinxupquote{fredpy.series}}}}} instance stores information about a FRED series in 17 attribues:
\begin{itemize}
\item {} 
\sphinxstylestrong{data:} (numpy ndarray) - data values.

\item {} 
\sphinxstylestrong{daterange:} (string) - specifies the dates of the first and last observations.

\item {} 
\sphinxstylestrong{dates:} (list) - list of date strings in YYYY-MM-DD format.

\item {} 
\sphinxstylestrong{datetimes:} (numpy ndarray) - array containing observation dates formatted as datetime objects.

\item {} 
\sphinxstylestrong{frequency:} (string) - data frequency. ‘Daily’, ‘Weekly’, ‘Monthly’, ‘Quarterly’, ‘Semiannual’, or ‘Annual’.

\item {} 
\sphinxstylestrong{frequency\_short:} (string) - data frequency. Abbreviated. ‘D’, ‘W’, ‘M’, ‘Q’, ‘SA, or ‘A’.

\item {} 
\sphinxstylestrong{last\_updated:} (string) - date series was last updated.

\item {} 
\sphinxstylestrong{notes:} (string) - details about series. Not available for all series.

\item {} 
\sphinxstylestrong{release:} (string) - statistical release containing data.

\item {} 
\sphinxstylestrong{seasonal\_adjustment:} (string) - specifies whether the data has been seasonally adjusted.

\item {} 
\sphinxstylestrong{seasonal\_adjustment\_short:} (string) - specifies whether the data has been seasonally adjusted. Abbreviated.

\item {} 
\sphinxstylestrong{series\_id:} (string) - unique FRED series ID code.

\item {} 
\sphinxstylestrong{source:} (string) - original source of the data.

\item {} 
\sphinxstylestrong{t:} (int) - number corresponding to frequency: 365 for daily, 52 for weekly, 12 for monthly, 4 for quarterly, and 1 for annual.

\item {} 
\sphinxstylestrong{title:} (string) - title of the data series.

\item {} 
\sphinxstylestrong{units:} (string) - units of the data series.

\item {} 
\sphinxstylestrong{units\_short:} (string) - units of the data series. Abbreviated.

\end{itemize}

\fvset{hllines={, ,}}%
\begin{Verbatim}[commandchars=\\\{\}]
\textcolor{nbsphinxin}{In [6]: }\PYG{c+c1}{\PYGZsh{} Print the title, the units, the frequency, the date range, and the source of the gdp data}
        \PYG{k}{print}\PYG{p}{(}\PYG{n}{gdp}\PYG{o}{.}\PYG{n}{title}\PYG{p}{)}
        \PYG{k}{print}\PYG{p}{(}\PYG{n}{gdp}\PYG{o}{.}\PYG{n}{units}\PYG{p}{)}
        \PYG{k}{print}\PYG{p}{(}\PYG{n}{gdp}\PYG{o}{.}\PYG{n}{frequency}\PYG{p}{)}
        \PYG{k}{print}\PYG{p}{(}\PYG{n}{gdp}\PYG{o}{.}\PYG{n}{daterange}\PYG{p}{)}
        \PYG{k}{print}\PYG{p}{(}\PYG{n}{gdp}\PYG{o}{.}\PYG{n}{source}\PYG{p}{)}
\end{Verbatim}



%
\begin{OriginalVerbatim}[commandchars=\\\{\}]
Real Gross Domestic Product
Billions of Chained 2012 Dollars
Quarterly
Range: 1947-01-01 to 2018-07-01
U.S. Bureau of Economic Analysis
\end{OriginalVerbatim}
% The following \relax is needed to avoid problems with adjacent ANSI
% cells and some other stuff (e.g. bullet lists) following ANSI cells.
% See https://github.com/sphinx-doc/sphinx/issues/3594
\relax

\fvset{hllines={, ,}}%
\begin{Verbatim}[commandchars=\\\{\}]
\textcolor{nbsphinxin}{In [7]: }\PYG{c+c1}{\PYGZsh{} Print the last 4 values of the gdp data}
        \PYG{k}{print}\PYG{p}{(}\PYG{n}{gdp}\PYG{o}{.}\PYG{n}{data}\PYG{p}{[}\PYG{o}{\PYGZhy{}}\PYG{l+m+mi}{4}\PYG{p}{:}\PYG{p}{]}\PYG{p}{,}\PYG{l+s+s1}{\PYGZsq{}}\PYG{l+s+se}{\PYGZbs{}n}\PYG{l+s+s1}{\PYGZsq{}}\PYG{p}{)}
        
        \PYG{c+c1}{\PYGZsh{} Print the last 4 values of the gdp series dates}
        \PYG{k}{print}\PYG{p}{(}\PYG{n}{gdp}\PYG{o}{.}\PYG{n}{dates}\PYG{p}{[}\PYG{o}{\PYGZhy{}}\PYG{l+m+mi}{4}\PYG{p}{:}\PYG{p}{]}\PYG{p}{,}\PYG{l+s+s1}{\PYGZsq{}}\PYG{l+s+se}{\PYGZbs{}n}\PYG{l+s+s1}{\PYGZsq{}}\PYG{p}{)}
        
        \PYG{c+c1}{\PYGZsh{} Print the last 4 values of the gdp series datetimes}
        \PYG{k}{print}\PYG{p}{(}\PYG{n}{gdp}\PYG{o}{.}\PYG{n}{datetimes}\PYG{p}{[}\PYG{o}{\PYGZhy{}}\PYG{l+m+mi}{4}\PYG{p}{:}\PYG{p}{]}\PYG{p}{)}
\end{Verbatim}



%
\begin{OriginalVerbatim}[commandchars=\\\{\}]
[18223.758 18323.963 18511.576 18671.497]

['2017-10-01', '2018-01-01', '2018-04-01', '2018-07-01']

[datetime.datetime(2017, 10, 1, 0, 0) datetime.datetime(2018, 1, 1, 0, 0)
 datetime.datetime(2018, 4, 1, 0, 0) datetime.datetime(2018, 7, 1, 0, 0)]
\end{OriginalVerbatim}
% The following \relax is needed to avoid problems with adjacent ANSI
% cells and some other stuff (e.g. bullet lists) following ANSI cells.
% See https://github.com/sphinx-doc/sphinx/issues/3594
\relax

\fvset{hllines={, ,}}%
\begin{Verbatim}[commandchars=\\\{\}]
\textcolor{nbsphinxin}{In [8]: }\PYG{c+c1}{\PYGZsh{} Plot real GDP data}
        \PYG{n}{fig} \PYG{o}{=} \PYG{n}{plt}\PYG{o}{.}\PYG{n}{figure}\PYG{p}{(}\PYG{p}{)}
        \PYG{n}{ax} \PYG{o}{=} \PYG{n}{fig}\PYG{o}{.}\PYG{n}{add\PYGZus{}subplot}\PYG{p}{(}\PYG{l+m+mi}{1}\PYG{p}{,}\PYG{l+m+mi}{1}\PYG{p}{,}\PYG{l+m+mi}{1}\PYG{p}{)}
        \PYG{n}{ax}\PYG{o}{.}\PYG{n}{plot\PYGZus{}date}\PYG{p}{(}\PYG{n}{gdp}\PYG{o}{.}\PYG{n}{datetimes}\PYG{p}{,}\PYG{n}{gdp}\PYG{o}{.}\PYG{n}{data}\PYG{p}{,}\PYG{l+s+s1}{\PYGZsq{}}\PYG{l+s+s1}{\PYGZhy{}}\PYG{l+s+s1}{\PYGZsq{}}\PYG{p}{,}\PYG{n}{lw}\PYG{o}{=}\PYG{l+m+mi}{3}\PYG{p}{,}\PYG{n}{alpha} \PYG{o}{=} \PYG{l+m+mf}{0.65}\PYG{p}{)}
        \PYG{n}{ax}\PYG{o}{.}\PYG{n}{grid}\PYG{p}{(}\PYG{p}{)}
        \PYG{n}{ax}\PYG{o}{.}\PYG{n}{set\PYGZus{}title}\PYG{p}{(}\PYG{n}{gdp}\PYG{o}{.}\PYG{n}{title}\PYG{p}{)}
        \PYG{n}{ax}\PYG{o}{.}\PYG{n}{set\PYGZus{}ylabel}\PYG{p}{(}\PYG{n}{gdp}\PYG{o}{.}\PYG{n}{units}\PYG{p}{)}
\end{Verbatim}

\fvset{hllines={, ,}}%
\begin{Verbatim}[commandchars=\\\{\}]
\textcolor{nbsphinxout}{Out[8]: }Text(0, 0.5, \PYGZsq{}Billions of Chained 2012 Dollars\PYGZsq{})
\end{Verbatim}

\noindent\sphinxincludegraphics{{fredpy_examples_14_1}.png}


\subsubsection{Methods}
\label{\detokenize{fredpy_examples:Methods}}
A {\hyperref[\detokenize{series_class:fredpy.series}]{\sphinxcrossref{\sphinxcode{\sphinxupquote{fredpy.series}}}}} instance has 22 methods:
\begin{itemize}
\item {} 
\sphinxstylestrong{apc}(log=True, method=’backward’)

\item {} 
\sphinxstylestrong{bpfilter}(low=6, high=32, K=12)

\item {} 
\sphinxstylestrong{cffilter}(low=6, high=32)

\item {} 
\sphinxstylestrong{copy}()

\item {} 
\sphinxstylestrong{divide}(series2)

\item {} 
\sphinxstylestrong{firstdiff}()

\item {} 
\sphinxstylestrong{hpfilter}(lamb=1600)

\item {} 
\sphinxstylestrong{lintrend}()

\item {} 
\sphinxstylestrong{log}()

\item {} 
\sphinxstylestrong{ma1side}(length)

\item {} 
\sphinxstylestrong{ma2side}(length)

\item {} 
\sphinxstylestrong{minus}(series2)

\item {} 
\sphinxstylestrong{monthtoannual}(method=’average’)

\item {} 
\sphinxstylestrong{monthtoquarter}(method=’average’)

\item {} 
\sphinxstylestrong{pc}(log=True, method=’backward’, annualized=False)

\item {} 
\sphinxstylestrong{percapita}(total\_pop=True)

\item {} 
\sphinxstylestrong{plus}(series2)

\item {} 
\sphinxstylestrong{quartertoannual}(method=’average’)

\item {} 
\sphinxstylestrong{recent}(N)

\item {} 
\sphinxstylestrong{recessions}(color=‘0.5’, alpha = 0.5)

\item {} 
\sphinxstylestrong{times}(series2)

\item {} 
\sphinxstylestrong{window}(win)

\end{itemize}

The fredpy documentation has detailed explanations of the use of these methods: \sphinxurl{http://www.briancjenkins.com/fredpy-package/documentation/build/html/series\_class.html}.

\fvset{hllines={, ,}}%
\begin{Verbatim}[commandchars=\\\{\}]
\textcolor{nbsphinxin}{In [9]: }\PYG{c+c1}{\PYGZsh{} Restrict GDP to observations from January 1, 1990 to present}
        \PYG{n}{win} \PYG{o}{=} \PYG{p}{[}\PYG{l+s+s1}{\PYGZsq{}}\PYG{l+s+s1}{01\PYGZhy{}01\PYGZhy{}1990}\PYG{l+s+s1}{\PYGZsq{}}\PYG{p}{,}\PYG{l+s+s1}{\PYGZsq{}}\PYG{l+s+s1}{01\PYGZhy{}01\PYGZhy{}2200}\PYG{l+s+s1}{\PYGZsq{}}\PYG{p}{]}
        \PYG{n}{gdp\PYGZus{}win} \PYG{o}{=} \PYG{n}{gdp}\PYG{o}{.}\PYG{n}{window}\PYG{p}{(}\PYG{n}{win}\PYG{p}{)}
        
        \PYG{c+c1}{\PYGZsh{} Plot}
        \PYG{n}{fig} \PYG{o}{=} \PYG{n}{plt}\PYG{o}{.}\PYG{n}{figure}\PYG{p}{(}\PYG{p}{)}
        \PYG{n}{ax} \PYG{o}{=} \PYG{n}{fig}\PYG{o}{.}\PYG{n}{add\PYGZus{}subplot}\PYG{p}{(}\PYG{l+m+mi}{1}\PYG{p}{,}\PYG{l+m+mi}{1}\PYG{p}{,}\PYG{l+m+mi}{1}\PYG{p}{)}
        \PYG{n}{ax}\PYG{o}{.}\PYG{n}{plot\PYGZus{}date}\PYG{p}{(}\PYG{n}{gdp\PYGZus{}win}\PYG{o}{.}\PYG{n}{datetimes}\PYG{p}{,}\PYG{n}{gdp\PYGZus{}win}\PYG{o}{.}\PYG{n}{data}\PYG{p}{,}\PYG{l+s+s1}{\PYGZsq{}}\PYG{l+s+s1}{\PYGZhy{}}\PYG{l+s+s1}{\PYGZsq{}}\PYG{p}{,}\PYG{n}{lw}\PYG{o}{=}\PYG{l+m+mi}{3}\PYG{p}{,}\PYG{n}{alpha} \PYG{o}{=} \PYG{l+m+mf}{0.65}\PYG{p}{)}
        \PYG{n}{ax}\PYG{o}{.}\PYG{n}{grid}\PYG{p}{(}\PYG{p}{)}
        \PYG{n}{ax}\PYG{o}{.}\PYG{n}{set\PYGZus{}title}\PYG{p}{(}\PYG{n}{gdp\PYGZus{}win}\PYG{o}{.}\PYG{n}{title}\PYG{p}{)}
        \PYG{n}{ax}\PYG{o}{.}\PYG{n}{set\PYGZus{}ylabel}\PYG{p}{(}\PYG{n}{gdp\PYGZus{}win}\PYG{o}{.}\PYG{n}{units}\PYG{p}{)}
        
        \PYG{c+c1}{\PYGZsh{} Plot recession bars}
        \PYG{n}{gdp\PYGZus{}win}\PYG{o}{.}\PYG{n}{recessions}\PYG{p}{(}\PYG{p}{)}
\end{Verbatim}

\noindent\sphinxincludegraphics{{fredpy_examples_16_0}.png}

\fvset{hllines={, ,}}%
\begin{Verbatim}[commandchars=\\\{\}]
\textcolor{nbsphinxin}{In [10]: }\PYG{c+c1}{\PYGZsh{} Compute and plot the (annualized) quarterly growth rate of real GDP}
         \PYG{n}{gdp\PYGZus{}pc} \PYG{o}{=} \PYG{n}{gdp}\PYG{o}{.}\PYG{n}{pc}\PYG{p}{(}\PYG{n}{annualized}\PYG{o}{=}\PYG{n+nb+bp}{True}\PYG{p}{)}
         
         \PYG{c+c1}{\PYGZsh{} Plot}
         \PYG{n}{fig} \PYG{o}{=} \PYG{n}{plt}\PYG{o}{.}\PYG{n}{figure}\PYG{p}{(}\PYG{p}{)}
         \PYG{n}{ax} \PYG{o}{=} \PYG{n}{fig}\PYG{o}{.}\PYG{n}{add\PYGZus{}subplot}\PYG{p}{(}\PYG{l+m+mi}{1}\PYG{p}{,}\PYG{l+m+mi}{1}\PYG{p}{,}\PYG{l+m+mi}{1}\PYG{p}{)}
         \PYG{n}{ax}\PYG{o}{.}\PYG{n}{plot\PYGZus{}date}\PYG{p}{(}\PYG{n}{gdp\PYGZus{}pc}\PYG{o}{.}\PYG{n}{datetimes}\PYG{p}{,}\PYG{n}{gdp\PYGZus{}pc}\PYG{o}{.}\PYG{n}{data}\PYG{p}{,}\PYG{l+s+s1}{\PYGZsq{}}\PYG{l+s+s1}{\PYGZhy{}}\PYG{l+s+s1}{\PYGZsq{}}\PYG{p}{,}\PYG{n}{lw}\PYG{o}{=}\PYG{l+m+mi}{3}\PYG{p}{,}\PYG{n}{alpha} \PYG{o}{=} \PYG{l+m+mf}{0.65}\PYG{p}{)}
         \PYG{n}{ax}\PYG{o}{.}\PYG{n}{grid}\PYG{p}{(}\PYG{p}{)}
         \PYG{n}{ax}\PYG{o}{.}\PYG{n}{set\PYGZus{}title}\PYG{p}{(}\PYG{n}{gdp\PYGZus{}pc}\PYG{o}{.}\PYG{n}{title}\PYG{p}{)}
         \PYG{n}{ax}\PYG{o}{.}\PYG{n}{set\PYGZus{}ylabel}\PYG{p}{(}\PYG{n}{gdp\PYGZus{}pc}\PYG{o}{.}\PYG{n}{units}\PYG{p}{)}
         
         \PYG{c+c1}{\PYGZsh{} Plot recession bars}
         \PYG{n}{gdp\PYGZus{}pc}\PYG{o}{.}\PYG{n}{recessions}\PYG{p}{(}\PYG{p}{)}
\end{Verbatim}

\noindent\sphinxincludegraphics{{fredpy_examples_17_0}.png}

\fvset{hllines={, ,}}%
\begin{Verbatim}[commandchars=\\\{\}]
\textcolor{nbsphinxin}{In [11]: }\PYG{c+c1}{\PYGZsh{} Compute and plot the log of real GDP}
         \PYG{n}{gdp\PYGZus{}log} \PYG{o}{=} \PYG{n}{gdp}\PYG{o}{.}\PYG{n}{log}\PYG{p}{(}\PYG{p}{)}
         
         \PYG{c+c1}{\PYGZsh{} Plot}
         \PYG{n}{fig} \PYG{o}{=} \PYG{n}{plt}\PYG{o}{.}\PYG{n}{figure}\PYG{p}{(}\PYG{p}{)}
         \PYG{n}{ax} \PYG{o}{=} \PYG{n}{fig}\PYG{o}{.}\PYG{n}{add\PYGZus{}subplot}\PYG{p}{(}\PYG{l+m+mi}{1}\PYG{p}{,}\PYG{l+m+mi}{1}\PYG{p}{,}\PYG{l+m+mi}{1}\PYG{p}{)}
         \PYG{n}{ax}\PYG{o}{.}\PYG{n}{plot\PYGZus{}date}\PYG{p}{(}\PYG{n}{gdp\PYGZus{}log}\PYG{o}{.}\PYG{n}{datetimes}\PYG{p}{,}\PYG{n}{gdp\PYGZus{}log}\PYG{o}{.}\PYG{n}{data}\PYG{p}{,}\PYG{l+s+s1}{\PYGZsq{}}\PYG{l+s+s1}{\PYGZhy{}}\PYG{l+s+s1}{\PYGZsq{}}\PYG{p}{,}\PYG{n}{lw}\PYG{o}{=}\PYG{l+m+mi}{3}\PYG{p}{,}\PYG{n}{alpha} \PYG{o}{=} \PYG{l+m+mf}{0.65}\PYG{p}{)}
         \PYG{n}{ax}\PYG{o}{.}\PYG{n}{set\PYGZus{}title}\PYG{p}{(}\PYG{n}{gdp\PYGZus{}log}\PYG{o}{.}\PYG{n}{title}\PYG{p}{)}
         \PYG{n}{ax}\PYG{o}{.}\PYG{n}{set\PYGZus{}ylabel}\PYG{p}{(}\PYG{n}{gdp\PYGZus{}log}\PYG{o}{.}\PYG{n}{units}\PYG{p}{)}
         \PYG{n}{ax}\PYG{o}{.}\PYG{n}{grid}\PYG{p}{(}\PYG{p}{)}
\end{Verbatim}

\noindent\sphinxincludegraphics{{fredpy_examples_18_0}.png}


\subsection{More examples}
\label{\detokenize{fredpy_examples:More-examples}}
The following examples demonstrate some additional \sphinxcode{\sphinxupquote{fredpy}}
functionality.


\subsubsection{Comparison of CPI and GDP deflator inflation}
\label{\detokenize{fredpy_examples:Comparison-of-CPI-and-GDP-deflator-inflation}}
CPI data are released monthly by the BLS while GDP deflator data are
released quarterly by the BEA. Here we’ll first convert the monthly CPI
data to monthly frequency compute inflation as the percentage change in
the respective index since on year prior.

\fvset{hllines={, ,}}%
\begin{Verbatim}[commandchars=\\\{\}]
\textcolor{nbsphinxin}{In [12]: }\PYG{c+c1}{\PYGZsh{} Download CPI and GDP deflator data}
         \PYG{n}{cpi} \PYG{o}{=} \PYG{n}{fp}\PYG{o}{.}\PYG{n}{series}\PYG{p}{(}\PYG{l+s+s1}{\PYGZsq{}}\PYG{l+s+s1}{CPIAUCSL}\PYG{l+s+s1}{\PYGZsq{}}\PYG{p}{)}
         \PYG{n}{deflator} \PYG{o}{=} \PYG{n}{fp}\PYG{o}{.}\PYG{n}{series}\PYG{p}{(}\PYG{l+s+s1}{\PYGZsq{}}\PYG{l+s+s1}{GDPDEF}\PYG{l+s+s1}{\PYGZsq{}}\PYG{p}{)}
         
         \PYG{n}{fig} \PYG{o}{=} \PYG{n}{plt}\PYG{o}{.}\PYG{n}{figure}\PYG{p}{(}\PYG{n}{figsize}\PYG{o}{=}\PYG{p}{(}\PYG{l+m+mi}{12}\PYG{p}{,}\PYG{l+m+mi}{4}\PYG{p}{)}\PYG{p}{)}
         \PYG{n}{ax} \PYG{o}{=} \PYG{n}{fig}\PYG{o}{.}\PYG{n}{add\PYGZus{}subplot}\PYG{p}{(}\PYG{l+m+mi}{1}\PYG{p}{,}\PYG{l+m+mi}{2}\PYG{p}{,}\PYG{l+m+mi}{1}\PYG{p}{)}
         \PYG{n}{ax}\PYG{o}{.}\PYG{n}{plot\PYGZus{}date}\PYG{p}{(}\PYG{n}{cpi}\PYG{o}{.}\PYG{n}{datetimes}\PYG{p}{,}\PYG{n}{cpi}\PYG{o}{.}\PYG{n}{data}\PYG{p}{,}\PYG{l+s+s1}{\PYGZsq{}}\PYG{l+s+s1}{\PYGZhy{}}\PYG{l+s+s1}{\PYGZsq{}}\PYG{p}{,}\PYG{n}{lw}\PYG{o}{=}\PYG{l+m+mi}{3}\PYG{p}{,}\PYG{n}{alpha} \PYG{o}{=} \PYG{l+m+mf}{0.65}\PYG{p}{)}
         \PYG{n}{ax}\PYG{o}{.}\PYG{n}{grid}\PYG{p}{(}\PYG{p}{)}
         \PYG{n}{ax}\PYG{o}{.}\PYG{n}{set\PYGZus{}title}\PYG{p}{(}\PYG{n}{cpi}\PYG{o}{.}\PYG{n}{title}\PYG{p}{)}
         \PYG{n}{ax}\PYG{o}{.}\PYG{n}{set\PYGZus{}ylabel}\PYG{p}{(}\PYG{n}{cpi}\PYG{o}{.}\PYG{n}{units}\PYG{p}{)}
         
         \PYG{n}{ax} \PYG{o}{=} \PYG{n}{fig}\PYG{o}{.}\PYG{n}{add\PYGZus{}subplot}\PYG{p}{(}\PYG{l+m+mi}{1}\PYG{p}{,}\PYG{l+m+mi}{2}\PYG{p}{,}\PYG{l+m+mi}{2}\PYG{p}{)}
         \PYG{n}{ax}\PYG{o}{.}\PYG{n}{plot\PYGZus{}date}\PYG{p}{(}\PYG{n}{deflator}\PYG{o}{.}\PYG{n}{datetimes}\PYG{p}{,}\PYG{n}{deflator}\PYG{o}{.}\PYG{n}{data}\PYG{p}{,}\PYG{l+s+s1}{\PYGZsq{}}\PYG{l+s+s1}{\PYGZhy{}m}\PYG{l+s+s1}{\PYGZsq{}}\PYG{p}{,}\PYG{n}{lw}\PYG{o}{=}\PYG{l+m+mi}{3}\PYG{p}{,}\PYG{n}{alpha} \PYG{o}{=} \PYG{l+m+mf}{0.65}\PYG{p}{)}
         \PYG{n}{ax}\PYG{o}{.}\PYG{n}{grid}\PYG{p}{(}\PYG{p}{)}
         \PYG{n}{ax}\PYG{o}{.}\PYG{n}{set\PYGZus{}title}\PYG{p}{(}\PYG{n}{deflator}\PYG{o}{.}\PYG{n}{title}\PYG{p}{)}
         \PYG{n}{ax}\PYG{o}{.}\PYG{n}{set\PYGZus{}ylabel}\PYG{p}{(}\PYG{n}{deflator}\PYG{o}{.}\PYG{n}{units}\PYG{p}{)}
\end{Verbatim}

\fvset{hllines={, ,}}%
\begin{Verbatim}[commandchars=\\\{\}]
\textcolor{nbsphinxout}{Out[12]: }Text(0, 0.5, \PYGZsq{}Index 2012=100\PYGZsq{})
\end{Verbatim}

\noindent\sphinxincludegraphics{{fredpy_examples_21_1}.png}

\fvset{hllines={, ,}}%
\begin{Verbatim}[commandchars=\\\{\}]
\textcolor{nbsphinxin}{In [13]: }\PYG{c+c1}{\PYGZsh{} The CPI data are produced at a monthly frequency}
         \PYG{k}{print}\PYG{p}{(}\PYG{n}{cpi}\PYG{o}{.}\PYG{n}{frequency}\PYG{p}{)}
         
         \PYG{c+c1}{\PYGZsh{} Convert CPI data to quarterly frequency to conform with the GDP deflator}
         \PYG{n}{cpi2} \PYG{o}{=} \PYG{n}{cpi}\PYG{o}{.}\PYG{n}{monthtoquarter}\PYG{p}{(}\PYG{p}{)}
         \PYG{k}{print}\PYG{p}{(}\PYG{n}{cpi2}\PYG{o}{.}\PYG{n}{frequency}\PYG{p}{)}
\end{Verbatim}



%
\begin{OriginalVerbatim}[commandchars=\\\{\}]
Monthly
Quarterly
\end{OriginalVerbatim}
% The following \relax is needed to avoid problems with adjacent ANSI
% cells and some other stuff (e.g. bullet lists) following ANSI cells.
% See https://github.com/sphinx-doc/sphinx/issues/3594
\relax

\fvset{hllines={, ,}}%
\begin{Verbatim}[commandchars=\\\{\}]
\textcolor{nbsphinxin}{In [14]: }\PYG{c+c1}{\PYGZsh{} Compute the inflation rate based on each index}
         \PYG{n}{cpi\PYGZus{}pi} \PYG{o}{=} \PYG{n}{cpi2}\PYG{o}{.}\PYG{n}{apc}\PYG{p}{(}\PYG{p}{)}
         \PYG{n}{def\PYGZus{}pi} \PYG{o}{=} \PYG{n}{deflator}\PYG{o}{.}\PYG{n}{apc}\PYG{p}{(}\PYG{p}{)}
         
         \PYG{c+c1}{\PYGZsh{} Print date ranges for new inflation series}
         \PYG{k}{print}\PYG{p}{(}\PYG{n}{cpi\PYGZus{}pi}\PYG{o}{.}\PYG{n}{daterange}\PYG{p}{)}
         \PYG{k}{print}\PYG{p}{(}\PYG{n}{def\PYGZus{}pi}\PYG{o}{.}\PYG{n}{daterange}\PYG{p}{)}
\end{Verbatim}



%
\begin{OriginalVerbatim}[commandchars=\\\{\}]
Range: 1948-01-01 to 2018-07-01
Range: 1948-01-01 to 2018-07-01
\end{OriginalVerbatim}
% The following \relax is needed to avoid problems with adjacent ANSI
% cells and some other stuff (e.g. bullet lists) following ANSI cells.
% See https://github.com/sphinx-doc/sphinx/issues/3594
\relax

\fvset{hllines={, ,}}%
\begin{Verbatim}[commandchars=\\\{\}]
\textcolor{nbsphinxin}{In [15]: }\PYG{n}{fig} \PYG{o}{=} \PYG{n}{plt}\PYG{o}{.}\PYG{n}{figure}\PYG{p}{(}\PYG{n}{figsize}\PYG{o}{=}\PYG{p}{(}\PYG{l+m+mi}{6}\PYG{p}{,}\PYG{l+m+mi}{4}\PYG{p}{)}\PYG{p}{)}
         \PYG{n}{ax} \PYG{o}{=} \PYG{n}{fig}\PYG{o}{.}\PYG{n}{add\PYGZus{}subplot}\PYG{p}{(}\PYG{l+m+mi}{1}\PYG{p}{,}\PYG{l+m+mi}{1}\PYG{p}{,}\PYG{l+m+mi}{1}\PYG{p}{)}
         \PYG{n}{ax}\PYG{o}{.}\PYG{n}{plot\PYGZus{}date}\PYG{p}{(}\PYG{n}{cpi\PYGZus{}pi}\PYG{o}{.}\PYG{n}{datetimes}\PYG{p}{,}\PYG{n}{cpi\PYGZus{}pi}\PYG{o}{.}\PYG{n}{data}\PYG{p}{,}\PYG{l+s+s1}{\PYGZsq{}}\PYG{l+s+s1}{\PYGZhy{}}\PYG{l+s+s1}{\PYGZsq{}}\PYG{p}{,}\PYG{n}{lw}\PYG{o}{=}\PYG{l+m+mi}{3}\PYG{p}{,}\PYG{n}{alpha} \PYG{o}{=} \PYG{l+m+mf}{0.65}\PYG{p}{,}\PYG{n}{label}\PYG{o}{=}\PYG{l+s+s1}{\PYGZsq{}}\PYG{l+s+s1}{cpi}\PYG{l+s+s1}{\PYGZsq{}}\PYG{p}{)}
         \PYG{n}{ax}\PYG{o}{.}\PYG{n}{plot\PYGZus{}date}\PYG{p}{(}\PYG{n}{def\PYGZus{}pi}\PYG{o}{.}\PYG{n}{datetimes}\PYG{p}{,}\PYG{n}{def\PYGZus{}pi}\PYG{o}{.}\PYG{n}{data}\PYG{p}{,}\PYG{l+s+s1}{\PYGZsq{}}\PYG{l+s+s1}{\PYGZhy{}}\PYG{l+s+s1}{\PYGZsq{}}\PYG{p}{,}\PYG{n}{lw}\PYG{o}{=}\PYG{l+m+mi}{3}\PYG{p}{,}\PYG{n}{alpha} \PYG{o}{=} \PYG{l+m+mf}{0.65}\PYG{p}{,}\PYG{n}{label}\PYG{o}{=}\PYG{l+s+s1}{\PYGZsq{}}\PYG{l+s+s1}{def}\PYG{l+s+s1}{\PYGZsq{}}\PYG{p}{)}
         \PYG{n}{ax}\PYG{o}{.}\PYG{n}{legend}\PYG{p}{(}\PYG{n}{loc}\PYG{o}{=}\PYG{l+s+s1}{\PYGZsq{}}\PYG{l+s+s1}{upper right}\PYG{l+s+s1}{\PYGZsq{}}\PYG{p}{)}
         \PYG{n}{ax}\PYG{o}{.}\PYG{n}{set\PYGZus{}title}\PYG{p}{(}\PYG{l+s+s1}{\PYGZsq{}}\PYG{l+s+s1}{US inflation}\PYG{l+s+s1}{\PYGZsq{}}\PYG{p}{)}
         \PYG{n}{ax}\PYG{o}{.}\PYG{n}{set\PYGZus{}ylabel}\PYG{p}{(}\PYG{l+s+s1}{\PYGZsq{}}\PYG{l+s+s1}{Percent}\PYG{l+s+s1}{\PYGZsq{}}\PYG{p}{)}
         \PYG{n}{ax}\PYG{o}{.}\PYG{n}{grid}\PYG{p}{(}\PYG{p}{)}
\end{Verbatim}

\noindent\sphinxincludegraphics{{fredpy_examples_24_0}.png}

Even though the CPI inflation rate is on average about .3\% higher the
GDP deflator inflation rate, the CPI and the GDP deflator produce
comparable measures of US inflation.


\subsubsection{Equalizing date ranges of different series}
\label{\detokenize{fredpy_examples:Equalizing-date-ranges-of-different-series}}
Often data series have different observation ranges. The {\hyperref[\detokenize{additional_functions:fredpy.window_equalize}]{\sphinxcrossref{\sphinxcode{\sphinxupquote{fredpy.window\_equalize()}}}}} function provides a quick way to set the date ranges for multiple series to the same interval.

\fvset{hllines={, ,}}%
\begin{Verbatim}[commandchars=\\\{\}]
\textcolor{nbsphinxin}{In [16]: }\PYG{c+c1}{\PYGZsh{} Download unemployment and 3 month T\PYGZhy{}bill data}
         \PYG{n}{unemp} \PYG{o}{=} \PYG{n}{fp}\PYG{o}{.}\PYG{n}{series}\PYG{p}{(}\PYG{l+s+s1}{\PYGZsq{}}\PYG{l+s+s1}{UNRATE}\PYG{l+s+s1}{\PYGZsq{}}\PYG{p}{)}
         \PYG{n}{tbill\PYGZus{}3m} \PYG{o}{=} \PYG{n}{fp}\PYG{o}{.}\PYG{n}{series}\PYG{p}{(}\PYG{l+s+s1}{\PYGZsq{}}\PYG{l+s+s1}{TB3MS}\PYG{l+s+s1}{\PYGZsq{}}\PYG{p}{)}
         
         \PYG{c+c1}{\PYGZsh{} Print date ranges for series}
         \PYG{k}{print}\PYG{p}{(}\PYG{n}{unemp}\PYG{o}{.}\PYG{n}{daterange}\PYG{p}{)}
         \PYG{k}{print}\PYG{p}{(}\PYG{n}{tbill\PYGZus{}3m}\PYG{o}{.}\PYG{n}{daterange}\PYG{p}{)}
         
         \PYG{c+c1}{\PYGZsh{} Equalize the date ranges}
         \PYG{n}{unemp}\PYG{p}{,} \PYG{n}{tbill\PYGZus{}3m} \PYG{o}{=} \PYG{n}{fp}\PYG{o}{.}\PYG{n}{window\PYGZus{}equalize}\PYG{p}{(}\PYG{p}{[}\PYG{n}{unemp}\PYG{p}{,} \PYG{n}{tbill\PYGZus{}3m}\PYG{p}{]}\PYG{p}{)}
         
         \PYG{c+c1}{\PYGZsh{} Print the new date ranges for series}
         \PYG{k}{print}\PYG{p}{(}\PYG{p}{)}
         \PYG{k}{print}\PYG{p}{(}\PYG{n}{unemp}\PYG{o}{.}\PYG{n}{daterange}\PYG{p}{)}
         \PYG{k}{print}\PYG{p}{(}\PYG{n}{tbill\PYGZus{}3m}\PYG{o}{.}\PYG{n}{daterange}\PYG{p}{)}
\end{Verbatim}



%
\begin{OriginalVerbatim}[commandchars=\\\{\}]
Range: 1948-01-01 to 2018-10-01
Range: 1934-01-01 to 2018-10-01

Range: 1948-01-01 to 2018-10-01
Range: 1948-01-01 to 2018-10-01
\end{OriginalVerbatim}
% The following \relax is needed to avoid problems with adjacent ANSI
% cells and some other stuff (e.g. bullet lists) following ANSI cells.
% See https://github.com/sphinx-doc/sphinx/issues/3594
\relax


\subsubsection{Filtering 1: Extracting business cycle components from quarterly data with the HP filter}
\label{\detokenize{fredpy_examples:Filtering-1:-Extracting-business-cycle-components-from-quarterly-data-with-the-HP-filter}}
\fvset{hllines={, ,}}%
\begin{Verbatim}[commandchars=\\\{\}]
\textcolor{nbsphinxin}{In [17]: }\PYG{c+c1}{\PYGZsh{} Download nominal GDP, the GDP deflator}
         
         \PYG{n}{gdp} \PYG{o}{=} \PYG{n}{fp}\PYG{o}{.}\PYG{n}{series}\PYG{p}{(}\PYG{l+s+s1}{\PYGZsq{}}\PYG{l+s+s1}{GDP}\PYG{l+s+s1}{\PYGZsq{}}\PYG{p}{)}
         \PYG{n}{defl} \PYG{o}{=} \PYG{n}{fp}\PYG{o}{.}\PYG{n}{series}\PYG{p}{(}\PYG{l+s+s1}{\PYGZsq{}}\PYG{l+s+s1}{GDPDEF}\PYG{l+s+s1}{\PYGZsq{}}\PYG{p}{)}
         
         \PYG{c+c1}{\PYGZsh{} Make sure that all series have the same window of observation}
         \PYG{n}{gdp}\PYG{p}{,}\PYG{n}{defl} \PYG{o}{=} \PYG{n}{fp}\PYG{o}{.}\PYG{n}{window\PYGZus{}equalize}\PYG{p}{(}\PYG{p}{[}\PYG{n}{gdp}\PYG{p}{,}\PYG{n}{defl}\PYG{p}{]}\PYG{p}{)}
         
         \PYG{c+c1}{\PYGZsh{} Deflate GDP series}
         \PYG{n}{gdp} \PYG{o}{=} \PYG{n}{gdp}\PYG{o}{.}\PYG{n}{divide}\PYG{p}{(}\PYG{n}{defl}\PYG{p}{)}
         
         \PYG{c+c1}{\PYGZsh{} Convert GDP to per capita terms}
         \PYG{n}{gdp} \PYG{o}{=} \PYG{n}{gdp}\PYG{o}{.}\PYG{n}{percapita}\PYG{p}{(}\PYG{p}{)}
         
         \PYG{c+c1}{\PYGZsh{} Take log of GDP}
         \PYG{n}{gdp} \PYG{o}{=} \PYG{n}{gdp}\PYG{o}{.}\PYG{n}{log}\PYG{p}{(}\PYG{p}{)}
\end{Verbatim}

\fvset{hllines={, ,}}%
\begin{Verbatim}[commandchars=\\\{\}]
\textcolor{nbsphinxin}{In [18]: }\PYG{c+c1}{\PYGZsh{} Plot log data}
         \PYG{n}{fig} \PYG{o}{=} \PYG{n}{plt}\PYG{o}{.}\PYG{n}{figure}\PYG{p}{(}\PYG{n}{figsize}\PYG{o}{=}\PYG{p}{(}\PYG{l+m+mi}{6}\PYG{p}{,}\PYG{l+m+mi}{4}\PYG{p}{)}\PYG{p}{)}
         
         \PYG{n}{ax1} \PYG{o}{=} \PYG{n}{fig}\PYG{o}{.}\PYG{n}{add\PYGZus{}subplot}\PYG{p}{(}\PYG{l+m+mi}{1}\PYG{p}{,}\PYG{l+m+mi}{1}\PYG{p}{,}\PYG{l+m+mi}{1}\PYG{p}{)}
         \PYG{n}{ax1}\PYG{o}{.}\PYG{n}{plot\PYGZus{}date}\PYG{p}{(}\PYG{n}{gdp}\PYG{o}{.}\PYG{n}{datetimes}\PYG{p}{,}\PYG{n}{gdp}\PYG{o}{.}\PYG{n}{data}\PYG{p}{,}\PYG{l+s+s1}{\PYGZsq{}}\PYG{l+s+s1}{\PYGZhy{}}\PYG{l+s+s1}{\PYGZsq{}}\PYG{p}{,}\PYG{n}{lw}\PYG{o}{=}\PYG{l+m+mi}{3}\PYG{p}{,}\PYG{n}{alpha} \PYG{o}{=} \PYG{l+m+mf}{0.65}\PYG{p}{)}
         \PYG{n}{ax1}\PYG{o}{.}\PYG{n}{grid}\PYG{p}{(}\PYG{p}{)}
         \PYG{n}{ax1}\PYG{o}{.}\PYG{n}{set\PYGZus{}title}\PYG{p}{(}\PYG{l+s+s1}{\PYGZsq{}}\PYG{l+s+s1}{log real GDP per capita}\PYG{l+s+s1}{\PYGZsq{}}\PYG{p}{)}
         \PYG{c+c1}{\PYGZsh{} ax1.set\PYGZus{}ylabel(gdp.units)}
         \PYG{n}{gdp}\PYG{o}{.}\PYG{n}{recessions}\PYG{p}{(}\PYG{p}{)}
         
         \PYG{n}{fig}\PYG{o}{.}\PYG{n}{tight\PYGZus{}layout}\PYG{p}{(}\PYG{p}{)}
\end{Verbatim}

\noindent\sphinxincludegraphics{{fredpy_examples_30_0}.png}

The post-Great Recession slowdown in US real GDP growth is apparent in
the figure.

\fvset{hllines={, ,}}%
\begin{Verbatim}[commandchars=\\\{\}]
\textcolor{nbsphinxin}{In [19]: }\PYG{c+c1}{\PYGZsh{} Compute the hpfilter}
         \PYG{n}{gdp\PYGZus{}cycle}\PYG{p}{,} \PYG{n}{gdp\PYGZus{}trend} \PYG{o}{=} \PYG{n}{gdp}\PYG{o}{.}\PYG{n}{hpfilter}\PYG{p}{(}\PYG{p}{)}
\end{Verbatim}

\fvset{hllines={, ,}}%
\begin{Verbatim}[commandchars=\\\{\}]
\textcolor{nbsphinxin}{In [20]: }\PYG{c+c1}{\PYGZsh{} Plot log data}
         \PYG{n}{fig} \PYG{o}{=} \PYG{n}{plt}\PYG{o}{.}\PYG{n}{figure}\PYG{p}{(}\PYG{n}{figsize}\PYG{o}{=}\PYG{p}{(}\PYG{l+m+mi}{6}\PYG{p}{,}\PYG{l+m+mi}{8}\PYG{p}{)}\PYG{p}{)}
         
         \PYG{n}{ax1} \PYG{o}{=} \PYG{n}{fig}\PYG{o}{.}\PYG{n}{add\PYGZus{}subplot}\PYG{p}{(}\PYG{l+m+mi}{2}\PYG{p}{,}\PYG{l+m+mi}{1}\PYG{p}{,}\PYG{l+m+mi}{1}\PYG{p}{)}
         \PYG{n}{ax1}\PYG{o}{.}\PYG{n}{plot\PYGZus{}date}\PYG{p}{(}\PYG{n}{gdp}\PYG{o}{.}\PYG{n}{datetimes}\PYG{p}{,}\PYG{n}{gdp}\PYG{o}{.}\PYG{n}{data}\PYG{p}{,}\PYG{l+s+s1}{\PYGZsq{}}\PYG{l+s+s1}{\PYGZhy{}}\PYG{l+s+s1}{\PYGZsq{}}\PYG{p}{,}\PYG{n}{lw}\PYG{o}{=}\PYG{l+m+mi}{3}\PYG{p}{,}\PYG{n}{alpha} \PYG{o}{=} \PYG{l+m+mf}{0.7}\PYG{p}{,}\PYG{n}{label}\PYG{o}{=}\PYG{l+s+s1}{\PYGZsq{}}\PYG{l+s+s1}{actual}\PYG{l+s+s1}{\PYGZsq{}}\PYG{p}{)}
         \PYG{n}{ax1}\PYG{o}{.}\PYG{n}{plot\PYGZus{}date}\PYG{p}{(}\PYG{n}{gdp\PYGZus{}trend}\PYG{o}{.}\PYG{n}{datetimes}\PYG{p}{,}\PYG{n}{gdp\PYGZus{}trend}\PYG{o}{.}\PYG{n}{data}\PYG{p}{,}\PYG{l+s+s1}{\PYGZsq{}}\PYG{l+s+s1}{r\PYGZhy{}}\PYG{l+s+s1}{\PYGZsq{}}\PYG{p}{,}\PYG{n}{lw}\PYG{o}{=}\PYG{l+m+mi}{3}\PYG{p}{,}\PYG{n}{alpha} \PYG{o}{=} \PYG{l+m+mf}{0.65}\PYG{p}{,}\PYG{n}{label}\PYG{o}{=}\PYG{l+s+s1}{\PYGZsq{}}\PYG{l+s+s1}{HP trend}\PYG{l+s+s1}{\PYGZsq{}}\PYG{p}{)}
         \PYG{n}{ax1}\PYG{o}{.}\PYG{n}{grid}\PYG{p}{(}\PYG{p}{)}
         \PYG{n}{ax1}\PYG{o}{.}\PYG{n}{set\PYGZus{}title}\PYG{p}{(}\PYG{l+s+s1}{\PYGZsq{}}\PYG{l+s+s1}{log real GDP per capita}\PYG{l+s+s1}{\PYGZsq{}}\PYG{p}{)}
         \PYG{n}{gdp}\PYG{o}{.}\PYG{n}{recessions}\PYG{p}{(}\PYG{p}{)}
         \PYG{n}{ax1}\PYG{o}{.}\PYG{n}{legend}\PYG{p}{(}\PYG{n}{loc}\PYG{o}{=}\PYG{l+s+s1}{\PYGZsq{}}\PYG{l+s+s1}{lower right}\PYG{l+s+s1}{\PYGZsq{}}\PYG{p}{)}
         \PYG{n}{fig}\PYG{o}{.}\PYG{n}{tight\PYGZus{}layout}\PYG{p}{(}\PYG{p}{)}
         
         \PYG{n}{ax1} \PYG{o}{=} \PYG{n}{fig}\PYG{o}{.}\PYG{n}{add\PYGZus{}subplot}\PYG{p}{(}\PYG{l+m+mi}{2}\PYG{p}{,}\PYG{l+m+mi}{1}\PYG{p}{,}\PYG{l+m+mi}{2}\PYG{p}{)}
         \PYG{n}{ax1}\PYG{o}{.}\PYG{n}{plot\PYGZus{}date}\PYG{p}{(}\PYG{n}{gdp\PYGZus{}cycle}\PYG{o}{.}\PYG{n}{datetimes}\PYG{p}{,}\PYG{n}{gdp\PYGZus{}cycle}\PYG{o}{.}\PYG{n}{data}\PYG{p}{,}\PYG{l+s+s1}{\PYGZsq{}}\PYG{l+s+s1}{b\PYGZhy{}}\PYG{l+s+s1}{\PYGZsq{}}\PYG{p}{,}\PYG{n}{lw}\PYG{o}{=}\PYG{l+m+mi}{3}\PYG{p}{,}\PYG{n}{alpha} \PYG{o}{=} \PYG{l+m+mf}{0.65}\PYG{p}{,}\PYG{n}{label}\PYG{o}{=}\PYG{l+s+s1}{\PYGZsq{}}\PYG{l+s+s1}{HP cycle}\PYG{l+s+s1}{\PYGZsq{}}\PYG{p}{)}
         \PYG{n}{ax1}\PYG{o}{.}\PYG{n}{grid}\PYG{p}{(}\PYG{p}{)}
         \PYG{n}{ax1}\PYG{o}{.}\PYG{n}{set\PYGZus{}title}\PYG{p}{(}\PYG{l+s+s1}{\PYGZsq{}}\PYG{l+s+s1}{log real GDP per capita \PYGZhy{} dev from trend}\PYG{l+s+s1}{\PYGZsq{}}\PYG{p}{)}
         \PYG{n}{gdp}\PYG{o}{.}\PYG{n}{recessions}\PYG{p}{(}\PYG{p}{)}
         \PYG{n}{ax1}\PYG{o}{.}\PYG{n}{legend}\PYG{p}{(}\PYG{n}{loc}\PYG{o}{=}\PYG{l+s+s1}{\PYGZsq{}}\PYG{l+s+s1}{lower right}\PYG{l+s+s1}{\PYGZsq{}}\PYG{p}{)}
         \PYG{n}{fig}\PYG{o}{.}\PYG{n}{tight\PYGZus{}layout}\PYG{p}{(}\PYG{p}{)}
\end{Verbatim}

\noindent\sphinxincludegraphics{{fredpy_examples_33_0}.png}


\subsubsection{Filtering 2: Extracting business cycle components from monthly data}
\label{\detokenize{fredpy_examples:Filtering-2:-Extracting-business-cycle-components-from-monthly-data}}
In Figure 1.5 from \sphinxstyleemphasis{The Conquest of American Inflation}, Thomas Sargent
compares the business cycle componenets (BP filtered) of monthly
inflation and unemployment data for the US from 1960-1982. Here we
replicate Figure 1.5 to include the most recently available data and we
also consturct the figure using HP filtered data.

\fvset{hllines={, ,}}%
\begin{Verbatim}[commandchars=\\\{\}]
\textcolor{nbsphinxin}{In [21]: }\PYG{n}{u} \PYG{o}{=} \PYG{n}{fp}\PYG{o}{.}\PYG{n}{series}\PYG{p}{(}\PYG{l+s+s1}{\PYGZsq{}}\PYG{l+s+s1}{LNS14000028}\PYG{l+s+s1}{\PYGZsq{}}\PYG{p}{)}
         \PYG{n}{p} \PYG{o}{=} \PYG{n}{fp}\PYG{o}{.}\PYG{n}{series}\PYG{p}{(}\PYG{l+s+s1}{\PYGZsq{}}\PYG{l+s+s1}{CPIAUCSL}\PYG{l+s+s1}{\PYGZsq{}}\PYG{p}{)}
         
         \PYG{c+c1}{\PYGZsh{} Construct the inflation series}
         \PYG{n}{p} \PYG{o}{=} \PYG{n}{p}\PYG{o}{.}\PYG{n}{pc}\PYG{p}{(}\PYG{n}{annualized}\PYG{o}{=}\PYG{n+nb+bp}{True}\PYG{p}{)}
         \PYG{n}{p} \PYG{o}{=} \PYG{n}{p}\PYG{o}{.}\PYG{n}{ma2side}\PYG{p}{(}\PYG{n}{length}\PYG{o}{=}\PYG{l+m+mi}{6}\PYG{p}{)}
         
         \PYG{c+c1}{\PYGZsh{} Make sure that the data inflation and unemployment series cver the same time interval}
         \PYG{n}{p}\PYG{p}{,}\PYG{n}{u} \PYG{o}{=} \PYG{n}{fp}\PYG{o}{.}\PYG{n}{window\PYGZus{}equalize}\PYG{p}{(}\PYG{p}{[}\PYG{n}{p}\PYG{p}{,}\PYG{n}{u}\PYG{p}{]}\PYG{p}{)}
         
         \PYG{c+c1}{\PYGZsh{} Data}
         
         \PYG{n}{fig} \PYG{o}{=} \PYG{n}{plt}\PYG{o}{.}\PYG{n}{figure}\PYG{p}{(}\PYG{p}{)}
         \PYG{n}{ax} \PYG{o}{=} \PYG{n}{fig}\PYG{o}{.}\PYG{n}{add\PYGZus{}subplot}\PYG{p}{(}\PYG{l+m+mi}{2}\PYG{p}{,}\PYG{l+m+mi}{1}\PYG{p}{,}\PYG{l+m+mi}{1}\PYG{p}{)}
         \PYG{n}{ax}\PYG{o}{.}\PYG{n}{plot\PYGZus{}date}\PYG{p}{(}\PYG{n}{u}\PYG{o}{.}\PYG{n}{datetimes}\PYG{p}{,}\PYG{n}{u}\PYG{o}{.}\PYG{n}{data}\PYG{p}{,}\PYG{l+s+s1}{\PYGZsq{}}\PYG{l+s+s1}{b\PYGZhy{}}\PYG{l+s+s1}{\PYGZsq{}}\PYG{p}{,}\PYG{n}{lw}\PYG{o}{=}\PYG{l+m+mi}{2}\PYG{p}{)}
         \PYG{n}{ax}\PYG{o}{.}\PYG{n}{grid}\PYG{p}{(}\PYG{n+nb+bp}{True}\PYG{p}{)}
         \PYG{n}{ax}\PYG{o}{.}\PYG{n}{set\PYGZus{}title}\PYG{p}{(}\PYG{l+s+s1}{\PYGZsq{}}\PYG{l+s+s1}{Inflation}\PYG{l+s+s1}{\PYGZsq{}}\PYG{p}{)}
         
         \PYG{n}{ax} \PYG{o}{=} \PYG{n}{fig}\PYG{o}{.}\PYG{n}{add\PYGZus{}subplot}\PYG{p}{(}\PYG{l+m+mi}{2}\PYG{p}{,}\PYG{l+m+mi}{1}\PYG{p}{,}\PYG{l+m+mi}{2}\PYG{p}{)}
         \PYG{n}{ax}\PYG{o}{.}\PYG{n}{plot\PYGZus{}date}\PYG{p}{(}\PYG{n}{p}\PYG{o}{.}\PYG{n}{datetimes}\PYG{p}{,}\PYG{n}{p}\PYG{o}{.}\PYG{n}{data}\PYG{p}{,}\PYG{l+s+s1}{\PYGZsq{}}\PYG{l+s+s1}{r\PYGZhy{}}\PYG{l+s+s1}{\PYGZsq{}}\PYG{p}{,}\PYG{n}{lw}\PYG{o}{=}\PYG{l+m+mi}{2}\PYG{p}{)}
         \PYG{n}{ax}\PYG{o}{.}\PYG{n}{grid}\PYG{p}{(}\PYG{n+nb+bp}{True}\PYG{p}{)}
         \PYG{n}{ax}\PYG{o}{.}\PYG{n}{set\PYGZus{}title}\PYG{p}{(}\PYG{l+s+s1}{\PYGZsq{}}\PYG{l+s+s1}{Unemployment}\PYG{l+s+s1}{\PYGZsq{}}\PYG{p}{)}
         
         \PYG{n}{fig}\PYG{o}{.}\PYG{n}{autofmt\PYGZus{}xdate}\PYG{p}{(}\PYG{p}{)}
\end{Verbatim}

\noindent\sphinxincludegraphics{{fredpy_examples_35_0}.png}

\fvset{hllines={, ,}}%
\begin{Verbatim}[commandchars=\\\{\}]
\textcolor{nbsphinxin}{In [22]: }\PYG{c+c1}{\PYGZsh{} Filter the data}
         \PYG{n}{p\PYGZus{}bpcycle}\PYG{p}{,}\PYG{n}{p\PYGZus{}bptrend} \PYG{o}{=} \PYG{n}{p}\PYG{o}{.}\PYG{n}{bpfilter}\PYG{p}{(}\PYG{n}{low}\PYG{o}{=}\PYG{l+m+mi}{24}\PYG{p}{,}\PYG{n}{high}\PYG{o}{=}\PYG{l+m+mi}{84}\PYG{p}{,}\PYG{n}{K}\PYG{o}{=}\PYG{l+m+mi}{84}\PYG{p}{)}
         \PYG{n}{u\PYGZus{}bpcycle}\PYG{p}{,}\PYG{n}{u\PYGZus{}bptrend} \PYG{o}{=} \PYG{n}{u}\PYG{o}{.}\PYG{n}{bpfilter}\PYG{p}{(}\PYG{n}{low}\PYG{o}{=}\PYG{l+m+mi}{24}\PYG{p}{,}\PYG{n}{high}\PYG{o}{=}\PYG{l+m+mi}{84}\PYG{p}{,}\PYG{n}{K}\PYG{o}{=}\PYG{l+m+mi}{84}\PYG{p}{)}
         
         \PYG{c+c1}{\PYGZsh{} Scatter plot of BP\PYGZhy{}filtered inflation and unemployment data (Sargent\PYGZsq{}s Figure 1.5)}
         \PYG{n}{fig} \PYG{o}{=} \PYG{n}{plt}\PYG{o}{.}\PYG{n}{figure}\PYG{p}{(}\PYG{p}{)}
         \PYG{n}{ax} \PYG{o}{=} \PYG{n}{fig}\PYG{o}{.}\PYG{n}{add\PYGZus{}subplot}\PYG{p}{(}\PYG{l+m+mi}{1}\PYG{p}{,}\PYG{l+m+mi}{1}\PYG{p}{,}\PYG{l+m+mi}{1}\PYG{p}{)}
         \PYG{n}{t} \PYG{o}{=} \PYG{n}{np}\PYG{o}{.}\PYG{n}{arange}\PYG{p}{(}\PYG{n+nb}{len}\PYG{p}{(}\PYG{n}{u\PYGZus{}bpcycle}\PYG{o}{.}\PYG{n}{data}\PYG{p}{)}\PYG{p}{)}
         \PYG{n}{ax}\PYG{o}{.}\PYG{n}{scatter}\PYG{p}{(}\PYG{n}{u\PYGZus{}bpcycle}\PYG{o}{.}\PYG{n}{data}\PYG{p}{,}\PYG{n}{p\PYGZus{}bpcycle}\PYG{o}{.}\PYG{n}{data}\PYG{p}{,}\PYG{n}{facecolors}\PYG{o}{=}\PYG{l+s+s1}{\PYGZsq{}}\PYG{l+s+s1}{none}\PYG{l+s+s1}{\PYGZsq{}}\PYG{p}{,}\PYG{n}{alpha}\PYG{o}{=}\PYG{l+m+mf}{0.75}\PYG{p}{,}\PYG{n}{s}\PYG{o}{=}\PYG{l+m+mi}{20}\PYG{p}{,}\PYG{n}{c}\PYG{o}{=}\PYG{n}{t}\PYG{p}{,} \PYG{n}{linewidths}\PYG{o}{=}\PYG{l+m+mf}{1.5}\PYG{p}{)}
         \PYG{n}{ax}\PYG{o}{.}\PYG{n}{set\PYGZus{}xlabel}\PYG{p}{(}\PYG{l+s+s1}{\PYGZsq{}}\PYG{l+s+s1}{unemployment rate (}\PYG{l+s+s1}{\PYGZpc{}}\PYG{l+s+s1}{)}\PYG{l+s+s1}{\PYGZsq{}}\PYG{p}{)}
         \PYG{n}{ax}\PYG{o}{.}\PYG{n}{set\PYGZus{}ylabel}\PYG{p}{(}\PYG{l+s+s1}{\PYGZsq{}}\PYG{l+s+s1}{inflation rate (}\PYG{l+s+s1}{\PYGZpc{}}\PYG{l+s+s1}{)}\PYG{l+s+s1}{\PYGZsq{}}\PYG{p}{)}
         \PYG{n}{ax}\PYG{o}{.}\PYG{n}{set\PYGZus{}title}\PYG{p}{(}\PYG{l+s+s1}{\PYGZsq{}}\PYG{l+s+s1}{Inflation and unemployment: BP\PYGZhy{}filtered data}\PYG{l+s+s1}{\PYGZsq{}}\PYG{p}{)}
         \PYG{n}{ax}\PYG{o}{.}\PYG{n}{grid}\PYG{p}{(}\PYG{n+nb+bp}{True}\PYG{p}{)}
\end{Verbatim}

\noindent\sphinxincludegraphics{{fredpy_examples_36_0}.png}

\fvset{hllines={, ,}}%
\begin{Verbatim}[commandchars=\\\{\}]
\textcolor{nbsphinxin}{In [23]: }\PYG{c+c1}{\PYGZsh{} HP filter}
         \PYG{n}{p\PYGZus{}hpcycle}\PYG{p}{,}\PYG{n}{p\PYGZus{}hptrend} \PYG{o}{=} \PYG{n}{p}\PYG{o}{.}\PYG{n}{hpfilter}\PYG{p}{(}\PYG{n}{lamb}\PYG{o}{=}\PYG{l+m+mi}{129600}\PYG{p}{)}
         \PYG{n}{u\PYGZus{}hpcycle}\PYG{p}{,}\PYG{n}{u\PYGZus{}hptrend} \PYG{o}{=} \PYG{n}{u}\PYG{o}{.}\PYG{n}{hpfilter}\PYG{p}{(}\PYG{n}{lamb}\PYG{o}{=}\PYG{l+m+mi}{129600}\PYG{p}{)}
         
         \PYG{c+c1}{\PYGZsh{} Scatter plot of BP\PYGZhy{}filtered inflation and unemployment data (Sargent\PYGZsq{}s Figure 1.5)}
         \PYG{n}{fig} \PYG{o}{=} \PYG{n}{plt}\PYG{o}{.}\PYG{n}{figure}\PYG{p}{(}\PYG{p}{)}
         \PYG{n}{ax} \PYG{o}{=} \PYG{n}{fig}\PYG{o}{.}\PYG{n}{add\PYGZus{}subplot}\PYG{p}{(}\PYG{l+m+mi}{1}\PYG{p}{,}\PYG{l+m+mi}{1}\PYG{p}{,}\PYG{l+m+mi}{1}\PYG{p}{)}
         \PYG{n}{t} \PYG{o}{=} \PYG{n}{np}\PYG{o}{.}\PYG{n}{arange}\PYG{p}{(}\PYG{n+nb}{len}\PYG{p}{(}\PYG{n}{u\PYGZus{}hpcycle}\PYG{o}{.}\PYG{n}{data}\PYG{p}{)}\PYG{p}{)}
         \PYG{n}{ax}\PYG{o}{.}\PYG{n}{scatter}\PYG{p}{(}\PYG{n}{u\PYGZus{}hpcycle}\PYG{o}{.}\PYG{n}{data}\PYG{p}{,}\PYG{n}{p\PYGZus{}hpcycle}\PYG{o}{.}\PYG{n}{data}\PYG{p}{,}\PYG{n}{facecolors}\PYG{o}{=}\PYG{l+s+s1}{\PYGZsq{}}\PYG{l+s+s1}{none}\PYG{l+s+s1}{\PYGZsq{}}\PYG{p}{,}\PYG{n}{alpha}\PYG{o}{=}\PYG{l+m+mf}{0.75}\PYG{p}{,}\PYG{n}{s}\PYG{o}{=}\PYG{l+m+mi}{20}\PYG{p}{,}\PYG{n}{c}\PYG{o}{=}\PYG{n}{t}\PYG{p}{,} \PYG{n}{linewidths}\PYG{o}{=}\PYG{l+m+mf}{1.5}\PYG{p}{)}
         \PYG{n}{ax}\PYG{o}{.}\PYG{n}{set\PYGZus{}xlabel}\PYG{p}{(}\PYG{l+s+s1}{\PYGZsq{}}\PYG{l+s+s1}{unemployment rate (}\PYG{l+s+s1}{\PYGZpc{}}\PYG{l+s+s1}{)}\PYG{l+s+s1}{\PYGZsq{}}\PYG{p}{)}
         \PYG{n}{ax}\PYG{o}{.}\PYG{n}{set\PYGZus{}ylabel}\PYG{p}{(}\PYG{l+s+s1}{\PYGZsq{}}\PYG{l+s+s1}{inflation rate (}\PYG{l+s+s1}{\PYGZpc{}}\PYG{l+s+s1}{)}\PYG{l+s+s1}{\PYGZsq{}}\PYG{p}{)}
         \PYG{n}{ax}\PYG{o}{.}\PYG{n}{set\PYGZus{}title}\PYG{p}{(}\PYG{l+s+s1}{\PYGZsq{}}\PYG{l+s+s1}{Inflation and unemployment: HP\PYGZhy{}filtered data}\PYG{l+s+s1}{\PYGZsq{}}\PYG{p}{)}
         \PYG{n}{ax}\PYG{o}{.}\PYG{n}{grid}\PYG{p}{(}\PYG{n+nb+bp}{True}\PYG{p}{)}
\end{Verbatim}

\noindent\sphinxincludegraphics{{fredpy_examples_37_0}.png}

The choice of filterning method appears to strongly influence the
results. While both filtering methods


\subsubsection{Exporting data sets}
\label{\detokenize{fredpy_examples:Exporting-data-sets}}
Exporting data inported with \sphinxcode{\sphinxupquote{fredpy}} to csv files is easy with
\sphinxcode{\sphinxupquote{Pandas}}.

\fvset{hllines={, ,}}%
\begin{Verbatim}[commandchars=\\\{\}]
\textcolor{nbsphinxin}{In [24]: }\PYG{c+c1}{\PYGZsh{} create a Pandas DataFrame}
         \PYG{n}{df} \PYG{o}{=} \PYG{n}{pd}\PYG{o}{.}\PYG{n}{DataFrame}\PYG{p}{(}\PYG{p}{\PYGZob{}}\PYG{l+s+s1}{\PYGZsq{}}\PYG{l+s+s1}{inflation}\PYG{l+s+s1}{\PYGZsq{}}\PYG{p}{:}\PYG{n}{p}\PYG{o}{.}\PYG{n}{data}\PYG{p}{,}
                             \PYG{l+s+s1}{\PYGZsq{}}\PYG{l+s+s1}{unemployment}\PYG{l+s+s1}{\PYGZsq{}}\PYG{p}{:}\PYG{n}{u}\PYG{o}{.}\PYG{n}{data}\PYG{p}{\PYGZcb{}}\PYG{p}{)}
         
         \PYG{c+c1}{\PYGZsh{} Set the index of the DataFrame}
         \PYG{n}{df} \PYG{o}{=} \PYG{n}{df}\PYG{o}{.}\PYG{n}{set\PYGZus{}index}\PYG{p}{(}\PYG{n}{pd}\PYG{o}{.}\PYG{n}{to\PYGZus{}datetime}\PYG{p}{(}\PYG{n}{p}\PYG{o}{.}\PYG{n}{dates}\PYG{p}{)}\PYG{p}{)}
         
         \PYG{k}{print}\PYG{p}{(}\PYG{n}{df}\PYG{o}{.}\PYG{n}{head}\PYG{p}{(}\PYG{p}{)}\PYG{p}{)}
         
         \PYG{c+c1}{\PYGZsh{} Export to csv}
         \PYG{n}{df}\PYG{o}{.}\PYG{n}{to\PYGZus{}csv}\PYG{p}{(}\PYG{l+s+s1}{\PYGZsq{}}\PYG{l+s+s1}{data.csv}\PYG{l+s+s1}{\PYGZsq{}}\PYG{p}{)}
\end{Verbatim}



%
\begin{OriginalVerbatim}[commandchars=\\\{\}]
               inflation  unemployment
1954-01-01  6.330313e-01           3.6
1954-02-01  2.609508e-01           3.8
1954-03-01 -1.411834e-14           4.1
1954-04-01 -2.979518e-01           4.7
1954-05-01 -8.570949e-01           4.6
\end{OriginalVerbatim}
% The following \relax is needed to avoid problems with adjacent ANSI
% cells and some other stuff (e.g. bullet lists) following ANSI cells.
% See https://github.com/sphinx-doc/sphinx/issues/3594
\relax


\chapter{Indices and tables}
\label{\detokenize{index:indices-and-tables}}\begin{itemize}
\item {} 
\DUrole{xref,std,std-ref}{genindex}

\item {} 
\DUrole{xref,std,std-ref}{search}

\end{itemize}



\renewcommand{\indexname}{Index}
\printindex
\end{document}